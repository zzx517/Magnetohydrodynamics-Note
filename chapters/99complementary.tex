
\chapter{\texorpdfstring{$\nabla$}{nabla} 算子}

\section{不同坐标系下的梯度、散度、旋度与拉普拉斯算子的表达式}

\subsection{直角坐标系}
\begin{equation}
\nabla = \pDeriv{}{x} \vector{e}_x
    + \pDeriv{}{y} \vector{e}_y
    + \pDeriv{}{z} \vector{e}_z
\end{equation}

\paragraph{梯度}
\begin{equation}
    \grad{f}
    = \pDeriv{f}{x} \vector{e}_x
    + \pDeriv{f}{y} \vector{e}_y
    + \pDeriv{f}{z} \vector{e}_z
\end{equation}

\paragraph{散度}
\begin{equation}
    \dive{\vector{A}}
    = \pDeriv{A_x}{x} + \pDeriv{A_y}{y} + \pDeriv{A_z}{z}
\end{equation}

\paragraph{旋度}
\begin{equation}
    \curl{\vector{A}}
    = \begin{vNiceMatrix}
        \vector{e}_x & \vector{e}_y & \vector{e}_z \\
        \pDerivd{}{x} & \pDerivd{}{y} & \pDerivd{}{z} \\
        A_x & A_y & A_z
    \end{vNiceMatrix}
\end{equation}

\paragraph{拉普拉斯算子}
\begin{equation}
    \laplace{f}
    = \pDerivS{f}{x} + \pDerivS{f}{y} + \pDerivS{f}{z}
\end{equation}

\subsection{柱坐标系}

\paragraph{梯度}
\begin{equation}
    \grad{f}
    = \pDeriv{f}{R} \vector{e}_R
    + \frac{1}{R} \pDeriv{f}{\varphi} \vector{e}_\varphi
    + \pDeriv{f}{Z} \vector{e}_Z
\end{equation}

\paragraph{散度}
\begin{equation}
    \dive{\vector{A}}
    = \frac{1}{R} \pDeriv{}{R} \big(R A_R\big)
    + \frac{1}{R} \pDeriv{A_\varphi}{\varphi}
    + \pDeriv{A_Z}{Z}
\end{equation}

\paragraph{旋度}
\begin{equation}
    \curl{\vector{A}}
    = \begin{vNiceMatrix}
        \dfrac{1}{R} \vector{e}_R & \vector{e}_\varphi & \dfrac{1}{R} \vector{e}_Z \\
        \pDerivd{}{R} & \pDerivd{}{\varphi} & \pDerivd{}{Z} \\
        A_R & R A_\varphi & A_Z
    \end{vNiceMatrix}
\end{equation}

\paragraph{拉普拉斯算子}
\begin{equation}
    \laplace{f}
    = \frac{1}{R} \pDeriv{}{R} \left(R \pDeriv{f}{R}\right)
    + \frac{1}{R^2} \pDerivS{f}{\varphi}
    + \pDerivS{f}{Z}
\end{equation}

\subsection{球坐标系}

\paragraph{梯度}
\begin{equation}
    \grad{f}
    = \pDeriv{f}{r} \vector{e}_r
    + \frac{1}{r} \pDeriv{f}{\theta} \vector{e}_\theta
    + \frac{1}{r \sin{\theta}} \pDeriv{f}{\varphi} \vector{e}_\varphi
\end{equation}

\paragraph{散度}
\begin{equation}
    \dive{\vector{A}}
    = \frac{1}{r^2} \pDeriv{}{r} \big(r^2 A_r\big)
    + \frac{1}{r \sin{\theta}} \pDeriv{}{\theta}
    \big(A_\theta \sin{\theta}\big)
    + \frac{1}{r \sin{\theta}} \pDeriv{A_\varphi}{\varphi}
\end{equation}

\paragraph{旋度}
\begin{equation}
    \curl{\vector{A}}
    = \frac{1}{r^2 \sin{\theta}}\begin{vNiceMatrix}
        \vector{e}_r & r \vector{e}_\theta & r \sin{\theta} \vector{e}_z \\
        \pDerivd{}{r} & \pDerivd{}{\theta} & \pDerivd{}{z} \\
        A_r & r A_\theta & r \sin{\theta} A_z
    \end{vNiceMatrix}
\end{equation}

\paragraph{拉普拉斯算子}
\begin{equation}
    \laplace{f}
    = \frac{1}{r^2} \pDeriv{}{r} \left(r^2 \pDeriv{f}{r}\right)
    + \frac{1}{r^2 \sin{\theta}} \pDeriv{}{\theta} \left(\sin{\theta} \pDeriv{f}{\\theta}\right)
    + \frac{1}{r^2 \sin^2{\theta}} \pDerivS{f}{z}
\end{equation}

\section{位矢 \texorpdfstring{$r$}{r} 及其衍生矢量}

\begin{align}
    \grad{r} &= \frac{\vector{r}}{r} = \hat{\vector{r}} \\
    \dive{\vector{r}} &= 3 \\
    \curl{\vector{r}} &= 0 \\
    \grad{\frac{1}{r}} &= - \frac{\vector{r}}{r^3} \\
    \laplace{\frac{1}{r}} &= - 4 \pi \delta(\vector{r}) \\
    \grad{r^2} &= 2 \vector{r} \\
    \dive{\frac{\vector{r}}{r^3}} &= 4 \pi \delta(\vector{r})
\end{align}

\section{\texorpdfstring{$\nabla$}{nabla} 算子的微分性与向量性}

\begin{align}
\grad{(f g)} &= g \grad{f} + f \grad{g} \\
\dive{(f \vector{A})} &= \grad{f} \cdot \vector{A} + f \dive{\vector{A}} \\
\curl{(f \vector{A})} &= \grad{f} \times \vector{A} + f \curl{\vector{A}} \\
\dive{(\vector{A} \times \vector{B})} &= (\curl{\vector{A}}) \cdot \vector{B} - (\curl{\vector{B}}) \cdot \vector{A} \\
\curl{(\vector{A} \times \vector{B})} &= (\vector{B} \cdot \grad) \vector{A} - (\dive{\vector{A}}) \vector{B} \\
& + (\dive{\vector{B}}) \vector{A} - (\vector{A} \cdot \grad) \vector{B} \\
\grad{(\vector{A} \cdot \vector{B})} &= \vector{A} \times (\curl{\vector{B}}) + (\vector{A} \cdot \grad) \vector{B} \\
& + \vector{B} \times (\curl{\vector{A}}) + (\vector{B} \cdot \grad) \vector{A} \\
\curl{(\curl{\vector{A}})} &= \grad{(\dive{\vector{A}})} - \nabla^2 \vector{A} \\
\dive{(\vector{A} \vector{B})} &= (\dive{\vector{A}}) \vector{B} + (\vector{A} \cdot \grad) \vector{B}
\end{align}

\section{\texorpdfstring{$\nabla$}{nabla} 算子的其他性质}

\begin{align}
    \curl{(\grad{f})} &= 0 \\
    \dive{(\curl{\vector{A}})} &= 0 \\
    \grad{f(u)} &= \grad{u} \Deriv{f}{u} \\
    \dive{\vector{A}(u)} &= \grad{u} \cdot \Deriv{\vector{A}}{u} \\
    \curl{\vector{A}(u)} &= \grad{u} \times \Deriv{\vector{A}}{u}
\end{align}

\section{相关定理}

高斯定理
\begin{equation}
    \oint_S \vector{A} \cdot \dif \vector{S}
    = \int_V \dive{\vector{A}} \, \dif V
\end{equation}

斯托克斯定理
\begin{equation}
    \oint_L \vector{A} \cdot \dif \vector{l}
    = \int_S \left( \dive{\vector{A}} \right) \cdot \dif \vector{S}
\end{equation}

格林公式
\begin{align}
    \oint_S \left( f \grad{g} \right) \cdot \dif \vector{S}
    &= \int_V \left(
        f \laplace{g} + \grad{f} \cdot \grad{g}
    \right) \, \dif V \\
    \oint_S \left( f \grad{g} - g \grad{f} \right) \cdot \dif \vector{S}
    &= \int_V \left(
        f \laplace{g} - g \laplace{f}
    \right) \, \dif V
\end{align}

\chapter{曲线坐标系基础}

一般性的曲线坐标系可以用一组坐标 $(q^1, q^2, q^3)$ 来表示。定义逆变(contravariant)基矢
\begin{equation}
    \vector{e}^i = \pDeriv{q^i}{\vector{r}} = \grad{q^i}
\end{equation}
以及协变(covariant)基矢
\begin{equation}
    \vector{e}_i = \pDeriv{\vector{r}}{q^i}
\end{equation}

逆变基矢与协变基矢互为互反向量
\begin{align}
    \vector{e}^i \cdot \vector{e}_j &= \delta^i_j
\end{align}
