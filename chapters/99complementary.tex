
\chapter{\texorpdfstring{$\grad$}{nabla} 算子}

\section{不同坐标系下的梯度、散度、旋度与拉普拉斯算子的表达式}

\subsection{直角坐标系}
\begin{equation}
\grad = \pdv{}{x} \vec{e}_x
    + \pdv{}{y} \vec{e}_y
    + \pdv{}{z} \vec{e}_z
\end{equation}

\paragraph{梯度}
\begin{equation}
    \grad{f}
    = \pdv{f}{x} \vec{e}_x
    + \pdv{f}{y} \vec{e}_y
    + \pdv{f}{z} \vec{e}_z
\end{equation}

\paragraph{散度}
\begin{equation}
    \dive{\vec{A}}
    = \pdv{A_x}{x} + \pdv{A_y}{y} + \pdv{A_z}{z}
\end{equation}

\paragraph{旋度}
\begin{equation}
    \curl{\vec{A}}
    = \begin{vNiceMatrix}
        \vec{e}_x & \vec{e}_y & \vec{e}_z \\
        \displaystyle\pdv{}{x} & \displaystyle\pdv{}{y} & \displaystyle\pdv{}{z} \\
        A_x & A_y & A_z
    \end{vNiceMatrix}
\end{equation}

\paragraph{拉普拉斯算子}
\begin{equation}
    \laplacian{f}
    = \pdv[2]{f}{x} + \pdv[2]{f}{y} + \pdv[2]{f}{z}
\end{equation}

\subsection{柱坐标系}

\paragraph{梯度}
\begin{equation}
    \grad{f}
    = \pdv{f}{R} \vec{e}_R
    + \frac{1}{R} \pdv{f}{\varphi} \vec{e}_\varphi
    + \pdv{f}{Z} \vec{e}_Z
\end{equation}

\paragraph{散度}
\begin{equation}
    \dive{\vec{A}}
    = \frac{1}{R} \pdv{}{R} \big(R A_R\big)
    + \frac{1}{R} \pdv{A_\varphi}{\varphi}
    + \pdv{A_Z}{Z}
\end{equation}

\paragraph{旋度}
\begin{equation}
    \curl{\vec{A}}
    = \begin{vNiceMatrix}
        \dfrac{1}{R} \vec{e}_R & \vec{e}_\varphi & \dfrac{1}{R} \vec{e}_Z \\
        \displaystyle\pdv{}{R} & \displaystyle\pdv{}{\varphi} & \displaystyle\pdv{}{Z} \\
        A_R & R A_\varphi & A_Z
    \end{vNiceMatrix}
\end{equation}

\paragraph{拉普拉斯算子}
\begin{equation}
    \laplacian{f}
    = \frac{1}{R} \pdv{}{R} \ab(R \pdv{f}{R})
    + \frac{1}{R^2} \pdv[2]{f}{\varphi}
    + \pdv[2]{f}{Z}
\end{equation}

\subsection{球坐标系}

\paragraph{梯度}
\begin{equation}
    \grad{f}
    = \pdv{f}{r} \vec{e}_r
    + \frac{1}{r} \pdv{f}{\theta} \vec{e}_\theta
    + \frac{1}{r \sin{\theta}} \pdv{f}{\varphi} \vec{e}_\varphi
\end{equation}

\paragraph{散度}
\begin{equation}
    \dive{\vec{A}}
    = \frac{1}{r^2} \pdv{}{r} \big(r^2 A_r\big)
    + \frac{1}{r \sin{\theta}} \pdv{}{\theta}
    \big(A_\theta \sin{\theta}\big)
    + \frac{1}{r \sin{\theta}} \pdv{A_\varphi}{\varphi}
\end{equation}

\paragraph{旋度}
\begin{equation}
    \curl{\vec{A}}
    = \frac{1}{r^2 \sin{\theta}}\begin{vNiceMatrix}
        \vec{e}_r & r \vec{e}_\theta & r \sin{\theta} \vec{e}_z \\
        \displaystyle\pdv{}{r} & \displaystyle\pdv{}{\theta} & \displaystyle\pdv{}{z} \\
        A_r & r A_\theta & r \sin{\theta} A_z
    \end{vNiceMatrix}
\end{equation}

\paragraph{拉普拉斯算子}
\begin{equation}
    \laplacian{f}
    = \frac{1}{r^2} \pdv{}{r} \ab(r^2 \pdv{f}{r})
    + \frac{1}{r^2 \sin{\theta}} \pdv{}{\theta} \ab(\sin{\theta} \pdv{f}{\\theta})
    + \frac{1}{r^2 \sin^2{\theta}} \pdv[2]{f}{z}
\end{equation}

\section{位矢 \texorpdfstring{$\vec{r}$}{r} 及其衍生矢量}

\begin{align}
    \grad{r} &= \frac{\vec{r}}{r} = \hat{\vec{r}} \\
    \dive{\vec{r}} &= 3 \\
    \curl{\vec{r}} &= 0 \\
    \grad{\frac{1}{r}} &= - \frac{\vec{r}}{r^3} \\
    \laplacian{\frac{1}{r}} &= - 4 \pi \delta(\vec{r}) \\
    \grad{r^2} &= 2 \vec{r} \\
    \dive{\frac{\vec{r}}{r^3}} &= 4 \pi \delta(\vec{r})
\end{align}

\section{\texorpdfstring{$\grad$}{nabla} 算子的微分性与向量性}

\begin{align}
\grad{(f g)} &= g \grad{f} + f \grad{g} \\
\dive{(f \vec{A})} &= \grad{f} \cdot \vec{A} + f \dive{\vec{A}} \\
\curl{(f \vec{A})} &= \grad{f} \times \vec{A} + f \curl{\vec{A}} \\
\dive{(\vec{A} \times \vec{B})} &= (\curl{\vec{A}}) \cdot \vec{B} - (\curl{\vec{B}}) \cdot \vec{A} \\
\curl{(\vec{A} \times \vec{B})} &= (\vec{B} \cdot \grad) \vec{A} - (\dive{\vec{A}}) \vec{B} \\
& + (\dive{\vec{B}}) \vec{A} - (\vec{A} \cdot \grad) \vec{B} \\
\grad{(\vec{A} \cdot \vec{B})} &= \vec{A} \times (\curl{\vec{B}}) + (\vec{A} \cdot \grad) \vec{B} \\
& + \vec{B} \times (\curl{\vec{A}}) + (\vec{B} \cdot \grad) \vec{A} \\
\curl{(\curl{\vec{A}})} &= \grad{(\dive{\vec{A}})} - \nabla^2 \vec{A} \\
\dive{(\vec{A} \vec{B})} &= (\dive{\vec{A}}) \vec{B} + (\vec{A} \cdot \grad) \vec{B}
\end{align}

\section{\texorpdfstring{$\grad$}{nabla} 算子的其他性质}

\begin{align}
    \curl\ab(\grad{f}) &= 0 \\
    \dive\ab(\curl\vec{A}) &= 0 \\
    \grad f(u) &= \grad{u} \odv{f}{u} \\
    \dive\vec{A}(u) &= \grad{u} \cdot \odv{\vec{A}}{u} \\
    \curl\vec{A}(u) &= \grad{u} \times \odv{\vec{A}}{u}
\end{align}

\section{相关定理}

高斯定理
\begin{equation}
    \oint_S \vec{A} \cdot \d \vec{S}
    = \int_V \dive{\vec{A}} \d V
\end{equation}

斯托克斯定理
\begin{equation}
    \oint_L \vec{A} \cdot \d \vec{l}
    = \int_S \ab( \dive{\vec{A}} ) \cdot \d \vec{S}
\end{equation}

格林公式
\begin{align}
    \oint_S \ab( f \grad{g} ) \cdot \d \vec{S}
    &= \int_V \ab(
        f \laplacian{g} + \grad{f} \cdot \grad{g}
    ) \d V \\
    \oint_S \ab( f \grad{g} - g \grad{f} ) \cdot \d \vec{S}
    &= \int_V \ab(
        f \laplacian{g} - g \laplacian{f}
    ) \d V
\end{align}

\chapter{曲线坐标系基础}

一般性的曲线坐标系可以用一组坐标 $(q^1, q^2, q^3)$ 来表示。定义逆变(contravariant)基矢
\begin{equation}
    \vec{e}^i = \pdv{q^i}{\vec{r}} = \grad{q^i}
\end{equation}
以及协变(covariant)基矢
\begin{equation}
    \vec{e}_i = \pdv{\vec{r}}{q^i}
\end{equation}

逆变基矢与协变基矢互为互反向量
\begin{align}
    \vec{e}^i \cdot \vec{e}_j &= \delta^i_j
\end{align}
