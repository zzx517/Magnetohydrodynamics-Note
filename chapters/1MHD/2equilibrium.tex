
\chapter{磁流体力学平衡}

在理想磁流体力学方程组中,令 $\pdv{}{t} = 0, \vec{u} = 0$,即可得到理想磁流体力学平衡方程组:
\begin{subequations}\begin{align}
    \vec{J} \times \vec{B} &= 0 \\
    \vec{J} &= \mu_0 \curl{\vec{B}} \\
    \dive{\vec{B}} &= 0
\end{align}\end{subequations}

消去电流密度 $\vec{J}$ ,则
\begin{equation}
    \grad{\ab( p + \frac{\vec{B}^2}{2 \mu_0} )} = \frac{1}{\mu_0} \vec{B} \cdot \grad{\vec{B}}
\end{equation}
等价于
\begin{equation}
    \nabla_\perp \ab( p + \frac{\vec{B}^2}{2 \mu_0} ) = \frac{\vec{B}^2}{\mu_0} \vec{\kappa}
\end{equation}
当磁力线为平行直线时,上式右边为零,则
\begin{equation}
    p + \frac{\vec{B}^2}{2 \mu_0} = \text{const} = \frac{\vec{B}_0^2}{2 \mu_0}
\end{equation}

\section{一维平衡问题}

\subsection{\texorpdfstring{$\theta$}{theta} 箍缩}

平衡量具有如下形式:
\begin{subequations}\begin{align}
\vec{B} &= B(\rho) \vec{e}_z \\
\vec{J} &= \mu_0 \pdv{B}{\rho} \vec{e}_\theta
\end{align}\end{subequations}

由于 $p + \frac{\vec{B}^2}{2 \mu_0} = \text{const}$,则外面($\rho$大)的磁场大,里面($\rho$小)的磁场小,形成磁阱,体现的是等离子体的抗磁性。

为了表示磁场约束的效率,引入比压 $\beta$ 的概念:
\begin{subequations}\begin{align}
\beta &= \frac{p}{B^2 / 2 \mu_0} \\
\text{或} \quad \beta' &= \frac{p}{p + B^2 / 2 \mu_0}
= \frac{\beta}{1 + \beta}
\end{align}\end{subequations}
通常我们认为 $\beta$ 越大,约束越好。
对于 $\theta$ 箍缩,有 $\beta'_{\max} \approx 1$ 或 $\beta_{\max} \approx + \infty$。

\subsection{\texorpdfstring{$Z$}{Z} 箍缩}

平衡量具有如下形式:
\begin{subequations}\begin{align}
\vec{B} &= B(\rho) \vec{e}_\theta \\
\vec{J} &= J(\rho) \vec{e}_z
\end{align}\end{subequations}
则平衡方程为
\begin{equation}
\grad{\ab( p + \frac{B^2}{2 \mu_0} )}
= \frac{B}{\mu_0 \rho} \pdv{(B \vec{e}_\theta)}{\theta}
\end{equation}
化简得
\begin{equation}
\odv{}{\rho} \ab( p + \frac{B^2}{2 \mu_0} )
+ \frac{B^2}{\mu_0 \rho} = 0
\end{equation}
这个平衡方程的解并不唯一。

\subsection{螺旋箍缩}

平衡量具有如下形式:
\begin{subequations}\begin{align}
\vec{B} &= B_\theta(\rho) \vec{e}_\theta + B_z(\rho) \vec{e}_z \\
\vec{J} &= J_\theta(\rho) \vec{e}_\theta + J_z(\rho) \vec{e}_z
\end{align}\end{subequations}
则平衡方程为
\begin{equation}
\odv{}{\rho} \ab( p + \frac{B_\theta^2 + B_z^2}{2 \mu_0} )
+ \frac{B_\theta^2}{\mu_0 \rho} = 0
\end{equation}
由于没有其他限制,所以 $B_\theta(\rho), B_z(\rho), p(\rho)$可以有多种选择,故螺旋箍缩可能有的平衡位形是很多的。

\section{二维平衡问题}

\subsection{Grad-Shafranov 方程}

在磁面坐标系 $(\rho, \theta, \zeta)$,下考察环向对称的托卡马克的平衡。其中,$\zeta$ 等于柱坐标下的环向角,有
\begin{subequations}\begin{align}
\grad{\zeta} &= \frac{1}{R} \hat{\vec{e}}_\zeta \\
\grad{\rho} \times \grad{\theta} &= \frac{R^2}{\Jacobian} \grad{\zeta} \\
\laplacian{\zeta} &= 0
\end{align}\end{subequations}
同时,由于轴对称,有
\begin{equation}
\grad{\zeta} \cdot \grad{} = 0
\end{equation}

托卡马克的轴对称平衡磁场可以写成
\begin{equation}
\vec{B} = I(\psi) \grad{\zeta}
+ \grad{\zeta} \times \grad{\psi}
\end{equation}
其中,$\psi = \psi(\rho)$ 为极向磁通。

易证,此形式的磁场自然满足平行于磁面的要求:
\begin{equation}\begin{aligned}
B^\rho &= \grad{\rho} \cdot \vec{B} \\
&= I \Metric^{\rho\zeta} \\
&= 0
\end{aligned}\end{equation}

电流密度为
\begin{equation}
\vec{J} = \frac{1}{\mu_0} \curl{\vec{B}} =
\frac{1}{\mu_0} \ab\big[
    \curl{\ab(\grad{\zeta} \times \grad{\psi})}
    - I' \ab(\grad{\zeta} \times \grad{\psi})
]
\end{equation}
% 其中
% \begin{equation}
% \curl{\ab(\grad{\zeta} \times \grad{\psi})}
% = \laplacian{\psi} \grad{\zeta}
% - \laplacian{\zeta} \grad{\psi}
% = \laplacian{\psi} \grad{\zeta}
% \end{equation}
% 则
% \begin{equation}
% \vec{J}
% = \mu_0 \laplacian{\psi} \grad{\zeta}
% - \mu_0 I' \ab(\grad{\zeta} \times \grad{\psi})
% \end{equation}

易证,此形式的电流密度自然满足平行于磁面的要求:
\begin{equation}\begin{aligned}
J^\rho &= \grad{\rho} \cdot \vec{J} \\
&= \frac{1}{\mu_0} \grad{\rho} \cdot \ab[\curl{\ab(\grad{\zeta} \times \grad{\psi})}] \\
&= \frac{1}{\mu_0} \dive{\ab[
    \ab(\grad{\zeta} \times \grad{\psi}) \times \grad{\rho}
]} \\
&= \frac{1}{\mu_0} \dive{\ab(
    \Metric^{\rho\zeta} \grad{\psi}
    - \psi' \Metric^{\rho\rho} \grad{\zeta}
)} \\
&= - \frac{1}{\mu_0} \grad{\zeta} \cdot \grad{\ab(
    \psi' \Metric^{\rho\rho}
)}
- \frac{\psi' \Metric^{\rho\rho}}{\mu_0} \laplacian{\zeta} \\
&= 0
\end{aligned}\end{equation}
同时,极向分量和环向分量分别为
\begin{equation}\begin{aligned}
J^\theta &= \grad{\theta} \cdot \vec{J} \\
&= \frac{1}{\mu_0} \grad{\theta} \cdot \ab[
    \curl{\ab(\grad{\zeta} \times \grad{\psi})}
]
- \frac{I'}{\mu_0} \grad{\theta} \cdot \ab(\grad{\zeta} \times \grad{\psi}) \\
&= \frac{1}{\mu_0} \dive{\ab[
    \ab(\grad{\zeta} \times \grad{\psi}) \times \grad{\theta}
]} - \frac{\psi' I'}{\mu_0 \Jacobian} \\
&= \frac{1}{\mu_0} \dive{\ab(
    \Metric^{\theta\zeta} \grad{\psi}
    - \psi' \Metric^{\rho\theta} \grad{\zeta}
)} - \frac{\psi' I'}{\mu_0 \Jacobian} \\
&=  - \frac{\psi' I'}{\mu_0 \Jacobian}
\end{aligned}\end{equation}
\begin{equation}\begin{aligned}
J^\zeta &= \grad{\zeta} \cdot \vec{J} \\
&= \frac{1}{\mu_0} \grad{\zeta}
\cdot \ab[\curl{\ab(\grad{\zeta} \times \grad{\psi})}] \\
&= \frac{1}{\mu_0} \dive{\ab[
    \ab(\grad{\zeta} \times \grad{\psi}) \times \grad{\zeta}
]} \\
&= \frac{1}{\mu_0} \dive{\ab(
    \Metric^{\zeta\zeta} \grad{\psi}
    - \psi' \Metric^{\rho\zeta} \grad{\zeta}
)} \\
&= \frac{1}{\mu_0} \dive{\ab(
    \frac{1}{R^2} \grad{\psi}
)}
\end{aligned}\end{equation}
则电流密度也可写成:
\begin{equation}\begin{aligned}
\vec{J} &= \Jacobian \ab(
    J^\theta \grad{\zeta} \times \grad{\rho}
    + J^\zeta \grad{\rho} \times \grad{\theta}
) \\
&= \Jacobian J^\theta \grad{\zeta} \times \grad{\rho}
+ R^2 J^\zeta \grad{\zeta} \\
&= \frac{\Jacobian}{\psi'} J^\theta \vec{B}
+ \ab(R^2 J^\zeta - \frac{\Jacobian I}{\psi'} J^\theta) \grad{\zeta}
\end{aligned}\end{equation}

则洛仑兹力
\begin{equation}\begin{aligned}
\vec{J} \times \vec{B}
&= \ab(R^2 J^\zeta - \frac{\Jacobian I}{\psi'} J^\theta)
\grad{\zeta} \times \vec{B} \\
&= \ab(R^2 \psi' J^\zeta - \Jacobian I J^\theta)
\grad{\zeta} \times \ab(\grad{\zeta} \times \grad{\rho}) \\
&= \ab(\frac{\Jacobian I}{R^2} J^\theta - \psi' J^\zeta) \grad{\rho}
\end{aligned}\end{equation}

代入 $\vec{J} \times \vec{B} = \grad{p}$ 得
\begin{equation}
\frac{\Jacobian I}{R^2} J^\theta - \psi' J^\zeta = p' \psi'
\end{equation}
化简得 Grad-Shafranov 方程
\begin{equation}
\mu_0 p' + \frac{I I'}{R^2} + \dive{\ab( \frac{1}{R^2} \grad{\psi} )} = 0
\end{equation}

只需要令 $\Metric^{\rho\theta} = 0, \Jacobian = r R$,以及 $R$ 为常数,即可得到一维圆柱平衡方程。

\subsection{G-S 方程的近轴展开}

以逆环径比 $\varepsilon = a / R \ll 1$ 作为小量,对 G-S 方程作微扰展开。

\subsection{G-S 方程的精确解}
