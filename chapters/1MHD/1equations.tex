
\chapter{磁流体力学方程组及其基本性质}

\section{磁流体力学方程组的建立}

\subsection{动理论描述}

\begin{equation}
    \Deriv{f_\alpha}{t} = \pDeriv{f_\alpha}{t} + \vector{v} \cdot \pDeriv{f_\alpha}{\vector{r}} + \Deriv{\vector{v}}{t} \cdot \pDeriv{f_\alpha}{\vector{v}} = \left( \pDeriv{f_\alpha}{t} \right)_c
\end{equation}

\begin{equation}
    \pDeriv{f_\alpha}{t} + \vector{v} \cdot \pDeriv{f_\alpha}{\vector{r}} + \frac{q_\alpha}{m_\alpha} \left( \vector{E} + \vector{v} \times \vector{B} \right) \cdot \pDeriv{f_\alpha}{\vector{v}} = \left( \pDeriv{f_\alpha}{t} \right)_c
\end{equation}

\subsection{矩方程}

\subsubsection{连续性方程}

\begin{equation}
    \int \left[
        \pDeriv{f_\alpha}{t} + \vector{v} \cdot \pDeriv{f_\alpha}{\vector{r}} + \frac{q_\alpha}{m_\alpha} \left( \vector{E} + \vector{v} \times \vector{B} \right) \cdot \pDeriv{f_\alpha}{\vector{v}}
    \right] \, \d^3 \vector{v}
    = \int \left( \pDeriv{f_\alpha}{t} \right)_c \, \d^3 \vector{v}
\end{equation}

第一项
\begin{equation}\begin{aligned}
    \int \pDeriv{f_\alpha}{t} \, \d^3 \vector{v}
    &= \pDeriv{}{t} \int f_\alpha \, \d^3 \vector{v} \\
    &= \pDeriv{n_\alpha}{t}
\end{aligned}\end{equation}

第二项
\begin{equation}\begin{aligned}
    \int \vector{v} \cdot \pDeriv{f_\alpha}{\vector{r}} \, \d^3 \vector{v}
    &= \pDeriv{}{\vector{r}} \cdot \int f_\alpha \vector{v} \, \d^3 \vector{v} \\
    &= \dive{(n_\alpha \vector{u}_\alpha)}
\end{aligned}\end{equation}

第三项
\begin{equation}\begin{aligned}
    \int \vector{E} \cdot \pDeriv{f_\alpha}{\vector{v}} \, \d^3 \vector{v}
    &= \int \pDeriv{}{\vector{v}} \cdot (f_\alpha \vector{E}) \, \d^3 \vector{v} \\
    &= \int_{S_{\vector{v}}} f_\alpha \vector{E} \cdot \d \vector{S}_{\vector{v}} \\
    &= 0
\end{aligned}\end{equation}
及
\begin{equation}\begin{aligned}
    \int \vector{v} \times \vector{B} \cdot \pDeriv{f_\alpha}{\vector{v}} \, \d^3 \vector{v}
    &= \int \pDeriv{}{\vector{v}} \cdot (f_\alpha \vector{v} \times \vector{B}) \, \d^3 \vector{v}
    - \int f_\alpha \pDeriv{}{\vector{v}} \cdot (\vector{v} \times \vector{B}) \, \d^3 \vector{v} \\
    &= \int_{S_{\vector{v}}} f_\alpha \vector{v} \times \vector{B} \cdot \d \vector{S}_{\vector{v}}
    - \int f_\alpha \cdot 0 \, \d^3 \vector{v} \\
    &= 0
\end{aligned}\end{equation}
其中
\begin{equation}
    \pDeriv{}{\vector{v}} \cdot (\vector{v} \times \vector{B})
    = \vector{B} \cdot \left( \pDeriv{}{\vector{v}} \times \vector{v} \right)
    - \vector{v} \cdot \left( \pDeriv{}{\vector{v}} \times \vector{B} \right)
    = 0
\end{equation}

方程右边
\begin{equation}
    \int \left( \pDeriv{f_\alpha}{t} \right)_c \, \d^3 \vector{v} = \left( \pDeriv{n_\alpha}{t} \right)_c
\end{equation}

故
\begin{equation}
    \pDeriv{n_\alpha}{t} + \dive{(n_\alpha \vector{u}_\alpha)} = \left( \pDeriv{n_\alpha}{t} \right)_c
\end{equation}

忽略非弹性碰撞
\begin{equation}\label{eq:Fluid_连续性方程}
    \pDeriv{n_\alpha}{t} + \dive{(n_\alpha \vector{u}_\alpha)} = 0
\end{equation}

\subsubsection{动量方程}

\begin{equation}
    m_\alpha \int \vector{v} \left[
        \pDeriv{f_\alpha}{t} + \vector{v} \cdot \pDeriv{f_\alpha}{\vector{r}} + \frac{q_\alpha}{m_\alpha} \left( \vector{E} + \vector{v} \times \vector{B} \right) \cdot \pDeriv{f_\alpha}{\vector{v}}
    \right] \, \d^3 \vector{v}
    = m_\alpha \int \vector{v} \left( \pDeriv{f_\alpha}{t} \right)_c \, \d^3 \vector{v}
\end{equation}

左一
\begin{equation}\begin{aligned}
    m_\alpha \int \vector{v} \pDeriv{f_\alpha}{t} \, \d^3 \vector{v}
    &= m_\alpha \pDeriv{}{t} \int \vector{v} f_\alpha \, \d^3 \vector{v} \\
    &= m_\alpha \pDeriv{\left( n_\alpha \vector{u}_\alpha \right)}{t}
\end{aligned}\end{equation}

左二
\begin{equation}\begin{aligned}
    m_\alpha \int \vector{v} \vector{v} \cdot \pDeriv{f_\alpha}{\vector{r}} \, \d^3 \vector{v}
    &= \pDeriv{}{\vector{r}} \cdot \int m_\alpha \vector{v} \vector{v} f_\alpha \, \d^3 \vector{v} \\
    &= \dive{\vector{P}_\alpha}
\end{aligned}\end{equation}
其中
\begin{equation}
    \vector{P}_\alpha = \int m_\alpha \vector{v} \vector{v} f_\alpha \, \d^3 \vector{v}
\end{equation}
叫做应力张量,也叫做动量通量。对比之下,压强张量的定义
\begin{equation}
    \vector{p}_\alpha = \int m_\alpha \vector{w}_\alpha \vector{w}_\alpha f_\alpha \, \d^3 \vector{v}
\end{equation}
其中 $\vector{w}_\alpha = \vector{v} - \vector{u}_\alpha$ 表示粒子的无规热运动速度,因此有
\begin{equation}
    \vector{P}_\alpha = \vector{p}_\alpha + m_\alpha n_\alpha \vector{u}_\alpha \vector{u}_\alpha
\end{equation}

粒子的标量压强则可以由压强张量直接给出
\begin{equation}
    p_\alpha = \frac{1}{3} \Tr\ab(\vector{p}_\alpha)
    = \frac{1}{3} \int m_\alpha \vector{w}_\alpha^2 f_\alpha \, \d^3 \vector{v}
\end{equation}
压强张量也可以写成
\begin{equation}
    \vector{p}_\alpha = p_\alpha \vector{I} + \vector{\chi}_\alpha
\end{equation}
其中,$\vector{\chi}_\alpha$ 为粘滞张量,代表了分布函数偏离球对称性的部分。

左三
\begin{equation}\begin{aligned}
&q_\alpha \int
    \vector{v} \left( \vector{E} + \vector{v} \times \vector{B} \right)
    \cdot \pDeriv{f_\alpha}{\vector{v}}
\, \d^3 \vector{v} \\
& \quad = q_\alpha \int \pDeriv{}{\vector{v}} \cdot \left[
    f_\alpha \left( \vector{E} + \vector{v} \times \vector{B} \right) \vector{v}
\right] \, \d^3 \vector{v} \\
& \quad \quad - q_\alpha \int f_\alpha \left[
    \vector{v} \pDeriv{}{\vector{v}} \cdot \left( \vector{E} + \vector{v} \times \vector{B} \right)
    + \left( \vector{E} + \vector{v} \times \vector{B} \right) \cdot \pDeriv{}{\vector{v}} \vector{v}
\right] \, \d^3 \vector{v} \\
& \quad = - q_\alpha \int f_\alpha \left( \vector{E} + \vector{v} \times \vector{B} \right) \, \d^3 \vector{v} \\
& \quad = - q_\alpha n_\alpha \left( \vector{E} + \vector{u}_\alpha \times \vector{B} \right)
\end{aligned}\end{equation}

定义摩擦力
\begin{equation}
    m_\alpha \int \vector{v} \left( \pDeriv{f_\alpha}{t} \right)_c \, \d^3 \vector{v}
    = \vector{F}_{r,\alpha}
\end{equation}
代表的是其他种类粒子对 $\alpha$ 类粒子碰撞带来的动量。

故
\begin{equation}\label{eq:Fluid_动量方程}
m_\alpha n_\alpha \left(
    \pDeriv{\vector{u}_\alpha}{t} + \vector{u}_\alpha \cdot \grad{\vector{u}_\alpha}
\right)
= - \grad{p_\alpha} - \dive{\vector{\chi}_\alpha} + q_\alpha n_\alpha \left( \vector{E} + \vector{u}_\alpha \times \vector{B} \right) + \vector{F}_{r,\alpha}
\end{equation}

\subsubsection{能量方程}

\begin{equation}
\int \frac{1}{2} m_\alpha \vector{v}^2 \left[
    \pDeriv{f_\alpha}{t} + \vector{v} \cdot \pDeriv{f_\alpha}{\vector{r}} + \frac{q_\alpha}{m_\alpha} \left( \vector{E} + \vector{v} \times \vector{B} \right) \cdot \pDeriv{f_\alpha}{\vector{v}}
\right] \, \d^3 \vector{v}
= \int \frac{1}{2} m_\alpha \vector{v}^2 \left( \pDeriv{f_\alpha}{t} \right)_c \, \d^3 \vector{v}
\end{equation}

左一
\begin{equation}\begin{aligned}
\int \frac{1}{2} m_\alpha \vector{v}^2 \pDeriv{f_\alpha}{t} \, \d^3 \vector{v}
&= \pDeriv{}{t} \int \frac{1}{2} m_\alpha \left(
    \vector{w}_\alpha^2 + 2 \vector{w}_\alpha \cdot \vector{u}_\alpha + \vector{u}_\alpha^2
\right) f_\alpha \, \d^3 \vector{v} \\
&= \pDeriv{}{t} \left(
    \frac{3}{2} p_\alpha + \frac{1}{2} m_\alpha n_\alpha \vector{u}_\alpha^2
\right)
\end{aligned}\end{equation}

定义能流密度
\begin{equation}
\vector{Q}_\alpha = \int \frac{1}{2} m_\alpha \vector{v}^2 \vector{v} f_\alpha \, \d^3 \vector{v}
\end{equation}
热流密度
\begin{equation}
\vector{q}_\alpha = \int \frac{1}{2} m_\alpha \vector{w}_\alpha^2 \vector{w}_\alpha f_\alpha \, \d^3 \vector{v}
\end{equation}
二者之间的变换关系
\begin{equation}
\vector{Q}_\alpha = \vector{q}_\alpha + \vector{u}_\alpha \cdot \vector{p}_\alpha
+ \frac{3}{2} p_\alpha \vector{u}_\alpha + \frac{1}{2} m_\alpha n_\alpha \vector{u}_\alpha^2 \vector{u}_\alpha
\end{equation}
则左二为
\begin{equation}\begin{aligned}
\int \frac{1}{2} m_\alpha \vector{v}^2 \vector{v} \cdot \pDeriv{f_\alpha}{\vector{r}} \, \d^3 \vector{v}
&= \dive{\vector{Q}_\alpha} \\
&= \dive{\vector{q}_\alpha} + \dive{(\vector{u}_\alpha \cdot \vector{p}_\alpha)}
+ \dive{\left[\left(\frac{1}{2} m_\alpha n_\alpha \vector{u}_\alpha^2 + \frac{3}{2} p_\alpha \right) \vector{u}_\alpha\right]}
\end{aligned}\end{equation}

左三
\begin{equation}\begin{aligned}
&\int \frac{1}{2} q_\alpha \vector{v}^2 \left( \vector{E} + \vector{v} \times \vector{B} \right) \cdot \pDeriv{f_\alpha}{\vector{v}} \, \d^3 \vector{v} \\
& \quad = \frac{1}{2} q_\alpha \int \pDeriv{}{\vector{v}} \cdot \left[
    f_\alpha \vector{v}^2 \left( \vector{E} + \vector{v} \times \vector{B} \right)
\right] \, \d^3 \vector{v} \\
&  \quad\quad - \frac{1}{2} q_\alpha \int f_\alpha \pDeriv{}{\vector{v}} \cdot \left[
    \vector{v}^2 \left( \vector{E} + \vector{v} \times \vector{B} \right)
\right] \, \d^3 \vector{v} \\
& \quad = - \frac{1}{2} q_\alpha \int f_\alpha \left[
    2 \vector{v} \cdot \left( \vector{E} + \vector{v} \times \vector{B} \right)
    + \vector{v}^2 \pDeriv{}{\vector{v}} \cdot \left( \vector{E} + \vector{v} \times \vector{B} \right)
\right] \, \d^3 \vector{v} \\
& \quad = - q_\alpha \int f_\alpha \vector{v} \cdot \vector{E} \, \d^3 \vector{v} \\
& \quad = - q_\alpha n_\alpha \vector{u}_\alpha \cdot \vector{E}
\end{aligned}\end{equation}

定义能量交换率或者是热量输运率密度
\begin{equation}
W_\alpha = \int \frac{1}{2} m_\alpha \vector{w}_\alpha^2 \left( \pDeriv{f_\alpha}{t} \right)_c \, \d^3 \vector{v}
\end{equation}
其代表了 $\alpha$ 类粒子与其他类粒子的热碰撞获得的热量。
则右一
\begin{equation}\begin{aligned}
\int \frac{1}{2} m_\alpha \vector{v}^2 \left( \pDeriv{f_\alpha}{t} \right)_c \, \d^3 \vector{v}
&= \int \frac{1}{2} m_\alpha \vector{w}_\alpha^2 \left( \pDeriv{f_\alpha}{t} \right)_c \, \d^3 \vector{v} \\
& \quad + \vector{u}_\alpha \cdot \int m_\alpha \vector{v} \left( \pDeriv{f_\alpha}{t} \right)_c \, \d^3 \vector{v} \\
& \quad - \frac{1}{2} m_\alpha \vector{u}_\alpha^2 \int \left( \pDeriv{f_\alpha}{t} \right)_c \, \d^3 \vector{v} \\
&= W_\alpha + \vector{u}_\alpha \cdot \vector{F}_{r,\alpha}
\end{aligned}\end{equation}

故
\begin{equation}\begin{aligned}\label{eq:Fluid_能量方程}
& \pDeriv{}{t} \left(
    \frac{1}{2} m_\alpha n_\alpha \vector{u}_\alpha^2 + \frac{3}{2} p_\alpha
\right)
+ \dive{\left[
    \left(\frac{1}{2} m_\alpha n_\alpha \vector{u}_\alpha^2
    + \frac{5}{2} p_\alpha \right) \vector{u}_\alpha
\right]} \\
& \quad = q_\alpha n_\alpha \vector{u}_\alpha \cdot \vector{E}
- \dive{\vector{q}_\alpha}
- \dive{(\vector{\chi}_\alpha \cdot \vector{u}_\alpha)}
+ W_\alpha + \vector{u}_\alpha \cdot \vector{F}_{r,\alpha}
\end{aligned}\end{equation}

\subsubsection{双流体方程组}

连续方程 \ref{eq:Fluid_连续性方程}、动量方程 \ref{eq:Fluid_动量方程} 和能量方程 \ref{eq:Fluid_能量方程} 就构成了标准的双(多)流体方程组
\begin{gather}
\pDeriv{n_\alpha}{t} + \dive{(n_\alpha \vector{u}_\alpha)} = 0
\tag{\ref{eq:Fluid_连续性方程}} \\
m_\alpha n_\alpha \left(
    \pDeriv{\vector{u}_\alpha}{t} + \vector{u}_\alpha \cdot \grad{\vector{u}_\alpha}
\right)
= - \grad{p_\alpha} - \dive{\vector{\chi}_\alpha} + q_\alpha n_\alpha \left( \vector{E} + \vector{u}_\alpha \times \vector{B} \right) + \vector{F}_{r,\alpha}
\tag{\ref{eq:Fluid_动量方程}} \\
\begin{aligned}
& \pDeriv{}{t} \left(
    \frac{1}{2} m_\alpha n_\alpha \vector{u}_\alpha^2 + \frac{3}{2} p_\alpha
\right)
+ \dive{\left[
    \left(\frac{1}{2} m_\alpha n_\alpha \vector{u}_\alpha^2
    + \frac{5}{2} p_\alpha \right) \vector{u}_\alpha
\right]} \\
& \quad = q_\alpha n_\alpha \vector{u}_\alpha \cdot \vector{E}
- \dive{\vector{q}_\alpha}
- \dive{(\vector{\chi}_\alpha \cdot \vector{u}_\alpha)}
+ W_\alpha + \vector{u}_\alpha \cdot \vector{F}_{r,\alpha}
\end{aligned}
\tag{\ref{eq:Fluid_能量方程}}
\end{gather}
其中,未知数为粒子的密度 $n_\alpha$、压强 $p_\alpha$ 和流速 $\vector{u}_\alpha$。
而粘滞张量 $\vector{\chi}_\alpha$、热流 $\vector{q}_\alpha$、碰撞摩擦力 $\vector{F}_{r,\alpha}$、热量输运率密度 $W_\alpha$ 与未知数之间的关系需要预先指定。

\subsection{从双流体方程组到磁流体力学方程组}

现在,我们考虑等离子体成分只有电子和离子两种。将离子版本和电子版本的连续性方程加起来,得到
\begin{equation}
\pDeriv{\rho}{t} + \dive{\left(\rho \vector{u}\right)} = 0
\end{equation}
其中,由于准中性条件 $n_e \approx n_i \approx n$,及 $m_i \gg m_e$,有
\begin{align}
\rho &= m_e n_e + m_i n_i \approx m_i n \\
\vector{u} &= \frac{1}{\rho} \left( m_e n_e \vector{u}_e + m_i n_i \vector{u}_i \right)
\approx \vector{u}_i
\end{align}
上述流体速度代表了质心的运动速度。由于离子质量远大于电子,单流体是集中考虑离子惯性效应的,电子惯性基本可以忽略。因此在离子尺度上的运动,电子都可以快速效应,使得等离子体整体保持电中性。当然,仍然存在小尺度和小量的电荷分离,这很小的电荷分离也能产生较为显著的电场,同时并不需要泊松方程来封闭系统。

将的离子版本和电子版本的动量方程加起来,忽略电子的惯性项,得到
\begin{equation}
\rho \Deriv{\vector{u}}{t} = \vector{J} \times \vector{B} - \dive{\vector{p}_\alpha}
\end{equation}
其中,等离子体的压强张量
\begin{equation}
    \vector{p} = \sum_\alpha \vector{p}_\alpha
\end{equation}
等离子体电流密度
\begin{equation}
    \vector{J} = \sum_\alpha q_\alpha n_\alpha \vector{u}_\alpha
\end{equation}
弹性碰撞摩擦力满足动量守恒
\begin{equation}
    \sum_\alpha \vector{F}_{r,\alpha} = 0
\end{equation}
由于准中性,电场项可以忽略
\begin{equation}
    \left( \sum_\alpha q_\alpha n_\alpha \right) \vector{E} \approx 0 \cdot \vector{E} = 0
\end{equation}

将的离子版本和电子版本的能量方程加起来,得到
\begin{equation}
\pDeriv{}{t} \left(
    \frac{1}{2} \rho \vector{u}^2 + \frac{3}{2} p
\right)
+ \dive{\left[\left(\frac{1}{2} \rho \vector{u}^2 + \frac{5}{2} p \right) \vector{u} \right]}
= \vector{J} \cdot \vector{E}
- \dive{\vector{q}} - \dive{(\vector{\chi} \cdot \vector{u})}
\end{equation}
其中
\begin{align}
\frac{1}{2} \rho \vector{u}^2 &= \sum_\alpha \frac{1}{2} m_\alpha n_\alpha \vector{u}_\alpha^2 \\
\vector{q} &= \sum_\alpha \vector{q}_\alpha \\
\vector{u} \cdot \vector{p} &= \sum_\alpha \vector{u}_\alpha \cdot \vector{p}_\alpha
\end{align}
并且由于弹性碰撞的能量守恒,因此碰撞引起的系统整体的热量输运可以忽略。

接下来考虑电子的动量方程,忽略电子惯性项,可得
\begin{equation}
\vector{E} + \vector{u} \times \vector{B} - \frac{\vector{J} \times \vector{B}}{e n} + \frac{\grad{p}}{2 e n} - \frac{\vector{F}_{r,e}}{e n} = 0
\end{equation}
其中,电子流速
\begin{equation}
\vector{u}_i = \frac{n_i}{n_e}\vector{u}_i - \frac{1}{e n_e} \vector{J} \approx \vector{u} - \frac{1}{e n} \vector{J}
\end{equation}
电子压强
\begin{equation}
    p_e \approx p_i \approx \frac{p}{2}
\end{equation}

摩擦力 $\vector{F}_{r,e}$ 代表了离子-电子间的弹性碰撞导致的电子动量损失。通过引入电子-离子平均碰撞频率 $\nu_{ei}$,我们可以将摩擦力写成如下形式
\begin{equation}
\vector{F}_{r,e} = \nu_{ie} n \frac{m_e m_i}{m_e + m_i}
\left( \vector{u}_i - \vector{u}_e \right)
\approx e n \eta \vector{J}
\end{equation}
其中,$\eta = \dfrac{\nu_{ie} m_e}{e^2 n}$ 为等离子体的电阻率。则电子的动量方程改写为
\begin{equation}
\vector{E} + \vector{u} \times \vector{B} - \frac{\vector{J} \times \vector{B}}{e n} + \frac{\grad{p}}{2 e n} - \eta \vector{J} = 0
\end{equation}
上式即是广义欧姆定律。第两项代表洛仑兹力,第三项表示霍尔效应,第四项是压强梯度,第五项是电阻效应。

接下来我们需要结合麦克斯韦方程组来估计广义欧姆定律中各项的量级。设场的空间特征长度为 $L$,时间特征尺度为 $T = \omega^{-1}$,则
\begin{subequations}\begin{align}
1 = \frac{|\curl{\vector{E}}|}{|\pderiv{\vector{B}}{t}|} &\sim \frac{E T}{B L} \sim \frac{E}{U B} \\
\frac{|\vector{u} \times \vector{B}|}{|\vector{E}|} &\sim \frac{u}{U} \\
\frac{|\vector{J} \times \vector{B}|}{e n |\vector{E}|} &\sim \frac{B^2}{\mu_0 L e n E} \sim \frac{\omega \omega_{ce}}{\omega_{pe}^2} \left(\frac{c}{U}\right)^2 \\
\frac{|\grad{p}|}{e n E} &\sim \frac{p}{L e n U B} = \frac{\omega}{\omega_{ci}} \left(\frac{c_s}{U}\right)^2
\end{align}\end{subequations}
其中,$\omega_{pe} = \sqrt{\dfrac{e^2 n}{\varepsilon_0 m_e}}$ 为电子等离子体频率,$\omega_{c\alpha} = \dfrac{e B}{m_\alpha}$ 为电子/离子的回旋频率,$c_s = \sqrt{\dfrac{p}{n m_i}}$ 为离子声速。当等离子体流速 $u \sim U$ 时,电场项与洛伦兹力处于同一量级。当等离子体的频率很小时($\omega \ll \omega_{pe}, \omega \ll \omega_{ci}$),霍尔效应和压强梯度项可以忽略。

考虑麦克斯韦方程组
\begin{subequations}\begin{align}
\dive{\vector{E}} &= \sum_\alpha q_\alpha n_\alpha \label{eq:电场泊松方程}\\
\curl{\vector{E}} &= - \pDeriv{\vector{B}}{t} \\
\dive{\vector{B}} &= 0 \\
\curl{\vector{B}} &= \mu_0 \vector{J} + \varepsilon_0 \mu_0 \pDeriv{\vector{E}}{t} \label{eq:磁场环路定理}
\end{align}\end{subequations}
其中,由于准中性,电场泊松方程 \ref{eq:电场泊松方程} 可以忽略。磁场环路定理 \ref{eq:磁场环路定理} 中 $\varepsilon_0 \mu_0 \pDeriv{\vector{E}}{t}$ 在不考虑高频电磁波的时候可以忽略。

故,磁流体力学方程组
\begin{subequations}\begin{align}
\pDeriv{\rho}{t} + \dive{\left(\rho \vector{u}\right)} &= 0 \\
\rho \Deriv{\vector{u}}{t} &= \vector{J} \times \vector{B} - \dive{\vector{p}_\alpha} \\
\pDeriv{}{t} \left(
    \frac{1}{2} \rho \vector{u}^2 + \frac{3}{2} p
\right)
+ \dive{\left[\left(\frac{1}{2} \rho \vector{u}^2 + \frac{5}{2} p \right) \vector{u} \right]}
&= \vector{J} \cdot \vector{E}
- \dive{\vector{q}} - \dive{(\vector{\chi} \cdot \vector{u})} \\
\vector{E} + \vector{u} \times \vector{B} &= \eta \vector{J} \\
\curl{\vector{E}} &= - \pDeriv{\vector{B}}{t} \\
\dive{\vector{B}} &= 0 \\
\curl{\vector{B}} &= \mu_0 \vector{J} \label{eq:02MHD_欧姆}
\end{align}\end{subequations}

\subsection{理想磁流体力学方程组}

首先忽略等离子体的粘滞张量
\begin{equation}
\vector{p} = p \vector{I}
\end{equation}

其次,在磁流体的 \textbf{低频} 近似下或者 \textbf{低碰撞} 条件下,可以忽略电阻效应。
进一步,不考虑热流函数 $\vector{q}$。
则动量方程、能量方程、广义欧姆定律简化为
\begin{gather}
\rho \Deriv{\vector{u}}{t} = \vector{J} \times \vector{B} - \grad{p} \\
\pDeriv{}{t} \left(
    \frac{1}{2} \rho \vector{u}^2 + \frac{3}{2} p
\right)
+ \dive{\left[\left(\frac{1}{2} \rho \vector{u}^2 + \frac{5}{2} \, p \right) \vector{u} \right]}
= \vector{J} \cdot \vector{E} \\
\vector{E} + \vector{u} \times \vector{B} = 0
\end{gather}

最后,利用热力学的一些定律之后,能量方程可以改写成如下形式
\begin{equation}
\Deriv{}{t} \left(\frac{p}{\rho^\gamma}\right) = 0 \tag{\ref{eq:IdealMHD_状态}}
\end{equation}

理想磁流体力学方程组
\begin{subequations}\begin{align}
\pDeriv{\rho}{t} + \dive{\left(\rho \vector{u}\right)} &= 0 \label{eq:IdealMHD_质量}\\
\rho \Deriv{\vector{u}}{t} &= \vector{J} \times \vector{B} - \grad{p} \label{eq:IdealMHD_动量}\\
\Deriv{}{t} \left(\frac{p}{\rho^\gamma}\right) &= 0 \label{eq:IdealMHD_状态}\\
\vector{E} + \vector{u} \times \vector{B} &= 0 \label{eq:IdealMHD_欧姆}\\
\curl{\vector{E}} &= - \pDeriv{\vector{B}}{t} \label{eq:IdealMHD_电场}\\
\dive{\vector{B}} &= 0 \label{eq:IdealMHD_磁场}\\
\curl{\vector{B}} &= \mu_0 \vector{J} \label{eq:IdealMHD_电流}
\end{align}\end{subequations}

\section{磁流体力学方程组的局限与适用范围}

\subsection{普通流体描述成立条件}

\paragraph{碰撞足够平频繁}
这就要求流体元的特征长度远大于中性粒子碰撞平均自由程
\begin{equation}
L \gg \lambda_c
\end{equation}
以及流体体元变化的特征时间远大于中性粒子碰撞平均时间
\begin{equation}
T = \omega^{-1} \gg \nu_c^{-1}
\end{equation}

\paragraph{热力学平衡态}
为了使流体元的速度能表征流体元中的绝大多数粒子的速度,
粒子的速度分布函数需要处于热力学平衡态 也即麦克斯韦分布。
\begin{equation}
f(\vector{v}) = \left(\frac{m}{2 \pi k T}\right)^{3/2}
\exp{\left(-\frac{m \vector{v}^2}{2 k T}\right)}
\end{equation}

\subsection{磁流体的特性}

\begin{enumerate}
    \item 带电粒子间的碰撞频率远大于中性粒子间的碰撞频率;
    \item 由于存在磁场,磁流体的特征长度在垂直磁场方向只需要远大于回旋半径就可以使用流体描述。
\end{enumerate}

因此等离子体、尤其是磁化等离子体,往往可以在比中性粒子体系更低的密度和更高的温度下仍能用流体方法来描述。

\subsection{磁流体描述成立条件}

\paragraph{准中性条件}
\begin{equation}
    n_e \approx n_i
\end{equation}
也即电子与离子的速度不能相差太大
\begin{equation}
    \vector{u}_e \approx \vector{u}_i
\end{equation}
但也允许少量的差距,对于自洽的电场和电流密度。

\paragraph{理想气体近似}
等离子体粒子热运动的特征动能需远大于粒子间平均库伦相互作用势能。
\begin{equation}
    T \gg \frac{e^2}{\varepsilon_0 d}
    \iff n \lambda_D^3 \gg 1
\end{equation}
等价于准中性条件,其中 $\lambda_D = \sqrt{\dfrac{\varepsilon_0 T}{e^2 n}}$ 为德拜长度。

\paragraph{无量子效应和相对论效应}

\section{理想磁流体力学方程组的基本性质}

\subsection{磁压力和磁张力}

洛伦兹力可以写成
\begin{equation}\begin{aligned}
    \vector{J} \times \vector{B} &= - \dive{\left(
    \frac{\vector{B}^2}{2 \mu_0} \vector{I} - \frac{\vector{B}\vector{B}}{\mu_0}
\right)} \\
&= \dive{\left(\frac{\vector{B}\vector{B}}{\mu_0}\right)}
- \grad{\left(\frac{\vector{B}^2}{2 \mu_0}\right)} \\
&= \frac{1}{\mu_0} B^2 \vector{\kappa}_B
- \frac{1}{2 \mu_0} \nabla_\perp \vector{B}^2
\end{aligned}\end{equation}
其中
\begin{equation}
\nabla_\perp = \nabla - \hat{\vector{b}} \hat{\vector{b}} \cdot \nabla
\end{equation}
为垂直磁场方向上的梯度。
\begin{equation}
\vector{\kappa}_B = \hat{\vector{b}} \cdot \grad{\hat{\vector{b}}}
\end{equation}
为磁力线的曲率。

\subsection{质量守恒}

连续性方程等价于守恒方程:
\begin{equation}
    \pDeriv{\rho}{t} + \dive{\left(\rho \vector{u}\right)} = 0
\iff\Deriv{}{t} \int_V \rho \, \d V = 0
\end{equation}

\subsection{动量守恒}

动量方程可以写成
\begin{equation}
\pDeriv{\rho \vector{u}}{t} + \dive{\left[
    \left(p + \frac{\vector{B}^2}{2 \mu_0}\right) \vector{I}
    + \rho \vector{u} \vector{u}
    - \frac{\vector{B} \vector{B}}{\mu_0}
\right]} = 0
\end{equation}

\subsection{能量守恒}

\begin{equation}\begin{aligned}
\vector{E} \cdot \vector{J} &= \frac{1}{\mu_0} (\curl{\vector{B}}) \cdot (\vector{B} \times \vector{u}) \\
&= \frac{1}{\mu_0} \dive{\Big[ \vector{B} \times \big(\vector{B} \times \vector{u} \big) \Big]}
    - \frac{1}{2 \mu_0} \pDeriv{\vector{B}^2}{t}
\end{aligned}\end{equation}

则能量方程可以写成
\begin{equation}
\pDeriv{}{t}\left(
    \frac{1}{2} \rho \vector{u}^2
    + \frac{3}{2} p
    + \frac{\vector{B}^2}{2 \mu_0}
\right)
+ \dive{\left[
    \left( \frac{1}{2} \rho \vector{u}^2 + \frac{5}{2} p  \right) \vector{u}
    + \frac{1}{\mu_0} \vector{B} \times \big( \vector{u} \times \vector{B} \big)
\right]} = 0
\end{equation}

\subsection{磁螺旋度守恒}

局部螺旋性 (helicity) 密度
\begin{equation}
    K = \vector{A} \cdot \vector{B}
\end{equation}
其中,$\vector{A}$ 为磁矢势。其与电标势 $\phi$ 有
\begin{align}
    \vector{E} &= - \grad{\phi} - \pDeriv{\vector{A}}{t} \\
    \vector{B} &= \curl{\vector{A}}
\end{align}

\begin{equation}\begin{aligned}
\pDeriv{K}{t} &= \vector{A} \cdot \pDeriv{\vector{B}}{t}
+ \vector{B} \cdot \pDeriv{\vector{A}}{t} \\
&= \dive{\left( \pDeriv{\vector{A}}{t} \times \vector{A} \right)}
+ 2 \vector{B} \cdot \pDeriv{\vector{A}}{t} \\
&= \dive{\left( \pDeriv{\vector{A}}{t} \times \vector{A} \right)}
- 2 \vector{B} \cdot \left(\grad{\phi} + \vector{E}\right)
\end{aligned}\end{equation}

则
\begin{equation}
\pDeriv{K}{t} + \dive{\left(
    \vector{A} \times \pDeriv{\vector{A}}{t} + 2 \phi \vector{B}
\right)} = - 2 \vector{E} \cdot \vector{B} = - 2 \eta \vector{J} \cdot \vector{B}
\end{equation}
对于理想 MHD ,忽略电阻项
\begin{equation}
\pDeriv{K}{t} + \dive{\left(
    \vector{A} \times \pDeriv{\vector{A}}{t} + 2 \phi \vector{B}
\right)} = 0
\end{equation}

由 $\phi = 0$ 及 $\vector{E} \times \d \vector{S} = 0$ 得
\begin{equation}
    \oint_S \left(
    \vector{A} \times \pDeriv{\vector{A}}{t} + 2 \phi \vector{B}
\right) \cdot \d \vector{S} = 0
\end{equation}
则
\begin{equation}
    \pDeriv{}{t} \int_V K \, \d V = 0
\end{equation}

另外,还有所谓的交叉螺旋性 (cross-helicity)
\begin{equation}
    K_c = \vector{u} \cdot \vector{B}
\end{equation}
在磁流体湍流中起到重要作用。对于不可压缩的理想磁流体来说,它也是一个守恒量。

\subsection{位力(virial)定理}

\subsubsection{经典力学}

考虑一个由质点构成的系统,质点 $m_i$ 的位置为 $\vector{r}_i$,作用其上的力为 $\vector{F}_i$,动力学方程为:
\begin{equation}
    m_i \Deriv{\vector{v}_i}{t} = \vector{F}_i
\end{equation}

考虑
\begin{equation}
    G = \sum_i m_i \vector{v}_i \cdot \vector{r}_i
\end{equation}
则
\begin{equation}\begin{aligned}
    \Deriv{G}{t} &= \sum_i \left( m_i \Deriv{\vector{v}_i}{t} \cdot \vector{r}_i
    + \sum_i m_i \vector{v}_i \cdot \Deriv{\vector{r}_i}{t} \right) \\
    &= 2 K + \sum_i \vector{F}_i \cdot \vector{r}_i
\end{aligned}\end{equation}
时间平均
\begin{equation}\begin{aligned}
    \frac{1}{T} \int_0^T \Deriv{G}{t} \, \d t
    &= 2 \left\langle K \right\rangle + \left\langle \sum_i \vector{F}_i \cdot \vector{r}_i \right\rangle \\
    &= \frac{G(T) - G(0)}{T}
\end{aligned}\end{equation}

当体系满足以下任意一个条件:
\begin{itemize}
    \item 体系是周期的,则在周期内,上式右边为零;
    \item 体系是有界的,则当 $T \to + \infty$ 时,上式右边为零。
\end{itemize}

若系统中质点的受力可以由一个标量势能函数 $V({\vector{r}_i})$ 导出,即
\begin{equation}
    \vector{F}_i = - \pDeriv{V}{\vector{r}_i}
\end{equation}
且体系的势能是是关于坐标的 $n$ 阶齐次函数:
\begin{equation}
    V({\alpha \vector{r}_i}) = \alpha^n V({\vector{r}_i})
\end{equation}
对 $\alpha$ 求导有
\begin{equation}\begin{aligned}
\pDeriv{V({\alpha \vector{r}_i})}{\alpha} &= \alpha \sum_i \pDeriv{V({\alpha \vector{r}_i})}{(\alpha \vector{r}_i)} \\
&= n \alpha^{n-1} V({\vector{r}_i})
\end{aligned}\end{equation}
令 $\alpha = 1$ 有
\begin{equation}
    \sum_i \pDeriv{V({\vector{r}_i})}{\vector{r}_i} = n V({\vector{r}_i})
\end{equation}

故有位力定理
\begin{equation}
    2 \langle K \rangle = n \langle V \rangle
\end{equation}

\subsubsection{量子力学}

待续

\subsubsection{磁流体}

\begin{equation}
\left\langle \int_V \left(
    \rho \vector{u}^2 + \frac{\vector{B}^2}{2 \mu_0} + 3 p + 3 \rho \phi_g
\right) \, \d V \right\rangle = 0
\end{equation}

\subsection{磁冻结与磁扩散}

考虑磁场随时间变化的规律
\begin{equation}\begin{aligned}
    \pDeriv{\vector{B}}{t} &= - \curl{\vector{E}} \\
    &= \curl{\big( \vector{u} \times \vector{B} - \eta \vector{J} \big)} \\
    &= \curl{\big( \vector{u} \times \vector{B} \big)}
    + \frac{\eta}{\mu_0} \laplace{\vector{B}}
    - \grad{\left(\frac{\eta}{\mu_0}\right)} \times \big(\curl{\vector{B}}\big)
\end{aligned}\end{equation}
一般有 $\dfrac{\eta}{\mu_0} = \text{const}$,则
\begin{equation}
    \pDeriv{\vector{B}}{t} = \curl{\big( \vector{u} \times \vector{B} \big)}
    + \frac{\eta}{\mu_0} \laplace{\vector{B}}
\end{equation}

\subsubsection{磁扩散}

当 $\vector{u}=0$ 时,有
\begin{equation}
    \pDeriv{\vector{B}}{t} = \frac{\eta}{\mu_0} \laplace{\vector{B}}
\end{equation}
这是一个扩散型方程, 表示磁场将从强场区向弱场区移动,而 $D_m = \dfrac{\eta}{\mu_0}$ 是磁扩散系数。

\subsubsection{磁冻结}

当电阻 $\eta \sim 0$ 时,有
\begin{equation}
    \pDeriv{\vector{B}}{t} = \curl{\big( \vector{u} \times \vector{B} \big)}
\end{equation}
其物理意义通常解释为磁力线的冻结,即磁力线与流体元一同运动。

\section{磁场的描述}

待续

\subsection{力线方程}

待续

\subsection{环形磁场与磁面坐标}

待续
