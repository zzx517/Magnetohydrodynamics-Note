
\chapter{磁流体力学波}

一个已经处在平衡态的磁流体体系,在受到小扰动时(外加扰动,内在扰动,湍动),往往会在平衡态附近作本征运动。它们被统称为(稳定的)磁流体力学波。本章将分别讨论均匀和非均匀磁流体中的典型磁流体波,同时拓展分析了托卡马克等离子体中一种特殊的声波。

求解本征值问题需要 \textbf{线性化}。即,把所有的量分成平衡量和扰动量两部分:$ f = f_0 + f_1 $。
一般来说,线性化的量级分析需要满足扰动量远小于平衡量,即 $f_0 \gg f_1$。
当然存在特殊情况,比如 $ \vec{u}_0 = \vec{0}, \vec{E}_0 = \vec{0} $,但是由于它们的平衡量不参与平衡,所以并不影响结论。
当扰动量增大到可以跟平衡量相比较的时候,非线性效应就显著起来了,线性化操作就不适用了。
扰动是时间的函数,但平衡量可以简单假定 $ \odv{f}!{t} = 0 $。

\section{均匀磁流体中的 MHD 波}

本节先考虑最简单的情况,平衡量都是均匀的:
$ \grad{p_0} = \vec{0}, \grad{\rho_0} = \vec{0}, \vec{B}_0 = B_0 \hat{\vec{b}}, \grad{B_0} = 0, \grad{\hat{\vec{b}}} = \vec{0} $。
并且扰动具有
$ f_1 \propto \exp{[\im (\vec{k} \cdot \vec{r} - \omega t)]} $ 的单色平面波形式。

\subsection{理想均匀磁流体中的 MHD 波}

基本方程是理想的 MHD 方程组:
\begin{align}
\pdv{\rho}{t} + \dive{\ab(\rho \vec{u})} &= 0
\tag{\ref{eq:IdealMHD_质量}} \\
\rho \mdv{\vec{u}}{t} &= \vec{J} \times \vec{B} - \grad{p}
\tag{\ref{eq:IdealMHD_动量}} \\
\mdv{}{t} \ab(\frac{p}{\rho^\gamma}) &= 0
\tag{\ref{eq:IdealMHD_状态}} \\
\vec{E} + \vec{u} \times \vec{B} &= 0
\tag{\ref{eq:IdealMHD_欧姆}} \\
\curl{\vec{E}} &= - \pdv{\vec{B}}{t}
\tag{\ref{eq:IdealMHD_电场}} \\
\dive{\vec{B}} &= 0
\tag{\ref{eq:IdealMHD_磁场}} \\
\curl{\vec{B}} &= \mu_0 \vec{J}
\tag{\ref{eq:IdealMHD_电流}}
\end{align}

接下来,将 MHD 方程组线性化:
\begin{subequations}\begin{align}
\pdv{\rho_1}{t} + \rho_0 \dive{\vec{u}_1} &= 0
\label{eq:线性IdealMHD_质量} \\
\rho_0 \pdv{\vec{u}_1}{t} &= \vec{J}_1 \times \vec{B}_0 - \grad{p_1}
\label{eq:线性IdealMHD_动量} \\
\pdv{p_1}{t} + \gamma p_0 \dive{\vec{u}_1} &= 0
\label{eq:线性IdealMHD_状态} \\
\vec{E}_1 + \vec{u}_1 \times \vec{B}_0 &= 0 \\
\curl{\vec{E}_1} &= - \pdv{\vec{B}_1}{t} \\
\dive{\vec{B}_1} &= 0 \\
\curl{\vec{B}_1} &= \mu_0 \vec{J}_1
\end{align}\end{subequations}
其中,电流密度的平衡量
\begin{equation}
\vec{J}_0 = \mu_0 \curl{\vec{B}_0} = \mu_0 \grad{B_0} \times \vec{e}_z = 0
\end{equation}

\paragraph{状态方程 \ref{eq:IdealMHD_状态} 的线性化}
首先,将全微分展开得到
\begin{equation}\begin{aligned}
\frac{1}{p} \mdv{p}{t} &= \frac{\gamma}{\rho} \mdv{\rho}{t} \\
\frac{1}{p} \ab(\pdv{p}{t} + \vec{u} \cdot \grad{p})
&= \frac{\gamma}{\rho} \ab(\pdv{\rho}{t} + \vec{u} \cdot \grad{\rho})
\end{aligned}\end{equation}

然后,线性化
\begin{equation}
\frac{1}{p_0} \pdv{p_1}{t}
= \frac{\gamma}{\rho_0} \pdv{\rho_1}{t}
\end{equation}
其中,分子的平衡量通过减去平衡方程去除,分母的扰动量根据 $f_0 \gg f_1$ 忽略。

最后,代入连续性方程的线性形式 \ref{eq:线性IdealMHD_质量} 消去密度项,得出状态方程的线性形式 \ref{eq:线性IdealMHD_状态}。

将线性的动量方程 \ref{eq:线性IdealMHD_动量} 再对时间 $t$ 求一次偏导,再代入其他扰动量得
\begin{equation}
\rho_0 \pdv[2]{\vec{u}_1}{t}
= \frac{1}{\mu_0} \bigg\{ \curl{\Big[ \curl{\big( \vec{u}_1 \times \vec{B}_0 \big)} \Big]} \bigg\} \times \vec{B}_0
+ \gamma p_0 \grad{\dive{\vec{u}_1}}
\end{equation}

对于具有单色平面波形式的扰动,有
\begin{subequations}\begin{align}
\pdv{f_1}{t} &= - \im \omega f_1 \\
\grad{f_1} &= \im \vec{k} f_1
\end{align}\end{subequations}
代入得
\begin{equation} \label{eq:04MHDwaves_raw}
\rho_0 \omega^2 \vec{u}_1
= \frac{1}{\mu_0} \bigg\{ \vec{k} \times \Big[ \vec{k} \times \big( \vec{u}_1 \times \vec{B}_0 \big) \Big] \bigg\} \times \vec{B}_0
+ \gamma p_0 \vec{k} \vec{k} \cdot \vec{u}_1
\end{equation}
这是一个形如 $ \vec{M} \cdot \vec{u}_1 = \vec{0} $ 的方程,
所以存在非平凡解的条件是系数矩阵 $\vec{M}$ 满足 $\det(\vec{M}) = 0$,
从而直接给出色散关系(dispersion relation)$\omega = \omega(\vec{k})$。

接下来展示使用矢量分析的办法来求解的过程。根据 $\vec{A} \times \big(\vec{B} \times \vec{C}\big)$,式 \ref{eq:04MHDwaves_raw} 变为
\begin{equation} \label{eq:04MHDwaves_raw2}
(\omega^2 - v_A^2 k^2) \vec{u}_1
= - v_A^2 \Big[ \vec{u}_1 \cdot \big(\vec{k} \times \hat{\vec{b}} \big) \Big] \vec{k} \times \hat{\vec{b}}
- v_A^2 k^2 \vec{u}_1 \cdot \hat{\vec{b}} \hat{\vec{b}}
+ c_s^2 \vec{u}_1 \cdot \vec{k} \vec{k}
\end{equation}
其中,$\hat{\vec{b}} = \dfrac{\vec{B}_0}{B_0} = \vec{e}_z$ 为磁场的方向,$c_s = \sqrt{\dfrac{\gamma p_0}{\rho_0}}$ 为等离子体声速,$v_A = \sqrt{\dfrac{B_0^2}{\mu_0 \rho_0}}$ 为阿尔芬速度。

用 $\vec{k}$ 点乘式 \ref{eq:04MHDwaves_raw2} 得
\begin{equation} \label{eq:04MHDwaves_k}
\big[\omega^2 - (v_A^2 + c_s^2) k^2 \big] \vec{u}_1 \cdot \vec{k}
+ v_A^2 k^2 \vec{u}_1 \cdot \hat{\vec{b}} \hat{\vec{b}} \cdot \vec{k} = 0
\end{equation}
带回式 \ref{eq:04MHDwaves_raw2} 得
\begin{equation}
(\omega^2 - v_A^2 k^2) \vec{u}_1
= - v_A^2 \Big[ \vec{u}_1 \cdot \big(\vec{k} \times \hat{\vec{b}} \big) \Big] \vec{k} \times \hat{\vec{b}}
- v_A^2 k^2 \vec{u}_1 \cdot \hat{\vec{b}} \hat{\vec{b}}
- \frac{v_A^2 c_s^2 k^2 \vec{u}_1 \cdot \hat{\vec{b}} \hat{\vec{b}} \cdot \vec{k}}{\omega^2 - (v_A^2 + c_s^2) k^2} \vec{k}
\end{equation}
再点乘式 $\hat{\vec{b}}$ 得
\begin{equation}
\ab[ \omega^2 + \frac{
    v_A^2 c_s^2 k^2 (\hat{\vec{b}} \cdot \vec{k})^2
}{\omega^2 - (v_A^2 + c_s^2) k^2} ] \big(\vec{u}_1 \cdot \hat{\vec{b}}\big) = 0
\end{equation}
上式有两个解
\begin{subequations}\begin{align}
\vec{u}_1 \cdot \hat{\vec{b}} &= 0
\label{eq:04MHDwaves_solution1}\\
\omega^2 + \frac{
    v_A^2 c_s^2 k^2 (\hat{\vec{b}} \cdot \vec{k})^2
}{\omega^2 - (v_A^2 + c_s^2) k^2} &= 0 \label{eq:04MHDwaves_solution2}
\end{align}\end{subequations}

用 $\vec{k} \times \hat{\vec{b}}$ 点乘式 \ref{eq:04MHDwaves_raw2} 得
\begin{equation}
\bigg\{\omega^2 + v_A^2 \Big[ \big(\vec{k} \times \hat{\vec{b}} \big)^2 - k^2\Big] \bigg\}
\Big[ \vec{u}_1 \cdot \big(\vec{k} \times \hat{\vec{b}} \big) \Big] = 0
\end{equation}
上式也有两个解
\begin{subequations}\begin{align}
\vec{u}_1 \cdot \big(\vec{k} \times \hat{\vec{b}} \big) &= 0
\label{eq:04MHDwaves_solution3} \\
\omega^2 + v_A^2 \Big[ \big(\vec{k} \times \hat{\vec{b}} \big)^2 - k^2\Big] &= 0
\label{eq:04MHDwaves_solution4}
\end{align}\end{subequations}

\subsubsection{剪切阿尔芬波}

考虑解 \ref{eq:04MHDwaves_solution1},式 \ref{eq:04MHDwaves_k} 变为
\begin{equation}
\big[\omega^2 - (v_A^2 + c_s^2) k^2 \big] \vec{u}_1 \cdot \vec{k} = 0
\end{equation}
上式有两解:
\begin{subequations}\begin{align}
\vec{u}_1 \cdot \vec{k} &= 0 \\
\omega^2 - (v_A^2 + c_s^2) k^2 &= 0
\end{align}\end{subequations}

而用 $\hat{\vec{b}}$ 点乘式 \ref{eq:04MHDwaves_raw2} 得
\begin{equation}
c_s^2 \vec{u}_1 \cdot \vec{k} \vec{k} \cdot \hat{\vec{b}} = 0
\end{equation}
上式也有两解:
\begin{subequations}\begin{align}
\vec{u}_1 \cdot \vec{k} &= 0 \\
\vec{k} \cdot \hat{\vec{b}} &= 0
\end{align}\end{subequations}

取两组解的共同部分 $\vec{u}_1 \cdot \vec{k} = 0$,即不可压缩条件 $\dive{\vec{u}_1} = 0$。

再考虑解 \ref{eq:04MHDwaves_solution1} 作为色散关系:
\begin{equation} \label{eq:04MHDwaves_SAW}
\omega^2 = k_\parallel^2 v_A^2
\end{equation}
这就是剪切阿尔芬波(Shear Alfvén Wave,SAW)的色散关系。其是等离子体中最低频的电磁波之一。
其中$k_\parallel = \vec{k} \cdot \hat{\vec{b}}$ 是平行波数,
有 $k^2 - (\vec{k} \times \hat{\vec{b}})^2 = (\vec{k} \cdot \hat{\vec{b}})^2$。

由于不可压缩条件 $\vec{u}_1 \cdot \vec{k} = 0$,剪切阿尔芬波不存在压强和密度扰动:
\begin{subequations}\begin{align}
\pdv{p_1}{t} &= - \gamma p_0 \dive{\vec{u}_1} = 0 \\
\pdv{\rho_1}{t} &= - \rho_0 \dive{\vec{u}_1} = 0
\end{align}\end{subequations}

从色散关系上看,$\omega \propto k_\parallel$,所以如果 $\vec{k} \perp \hat{\vec{b}}$,则无法形成 SAW。当 $\vec{k} \parallel \hat{\vec{b}}$ 时,上述推导过程不成立,因为色散关系的导出利用了 $\vec{u}_1 \cdot \big(\vec{k} \times \hat{\vec{b}} \big) \neq 0$ 的条件。但当 $\vec{k} \propto \hat{\vec{b}}$ 时,从式 \ref{eq:04MHDwaves_raw2} 也可得到 $\omega^2 = k^2 v_A^2$,与式 \ref{eq:04MHDwaves_SAW} 形式一致。

对于电场扰动 $\vec{E}_1$ 和磁场扰动 $\vec{B}_1$,有
\begin{subequations}\begin{align}
\vec{E}_1 &= B_0 \hat{\vec{b}} \times \vec{u}_1 \\
\vec{B}_1 &= - B_0 \frac{k_\parallel}{\omega} \vec{u}_1
\end{align}\end{subequations}

因此对 SAW,我们可以综合给出:
\begin{subequations}\begin{align}
\vec{u}_1 &\perp \vec{k} \rightarrow \text{流体横波} \\
\vec{u}_1 &\perp \hat{\vec{b}} \\
\vec{u}_1 &\parallel \vec{B}_1 \\
\vec{B}_1 &\perp \vec{k} \rightarrow \text{磁横波} \\
\vec{B}_1 &\perp \hat{\vec{b}}
\end{align}\end{subequations}
所以剪切阿尔芬波是等离子体横波 + 磁横波。因此,若只考虑剪切阿尔芬波,可以令 $p_1=0$、$\rho_1=0$、$\vec{k}\parallel\hat{\vec{b}}$ 和 $\vec{u}_1 \cdot \vec{k}=0$。

\subsubsection{磁声波}

考虑解 \ref{eq:04MHDwaves_solution2},得
\begin{equation}
\omega^2 = \frac{v_A^2 + c_s^2}{2} k^2 \ab[
    1 \pm \sqrt{1 - \frac{
        4 v_A^2 c_s^2
    }{
        (v_A^2 + c_s^2)^2
    }
    \frac{k_\parallel^2}{k^2} }
]
\end{equation}
此即磁声波(Magnetosonic waves)的色散关系,$\pm$ 号分别对应快/慢磁声波。

\paragraph{
    $\vec{k} \parallel \hat{\vec{b}}$
    的情况
}

色散方程可化简为
\begin{equation}
\omega^2 = \begin{cases}
    k^2 v_A^2 \\
    k^2 c_s^2
\end{cases}
\end{equation}
这表明:快(慢)磁声波在 $\vec{k} \parallel \hat{\vec{b}}$ 的条件下,存在唯一一个非平凡解,即声波,另一支与阿尔芬波重复。

对于声波来说,有
\begin{subequations}\begin{align}
\vec{k} &\parallel \hat{\vec{b}} \\
\vec{u}_1 &\cdot \vec{k} \neq 0 \rightarrow \text{流体纵波}
\end{align}\end{subequations}

\paragraph{
    $\vec{k} \perp \hat{\vec{b}}$
    的情况
}

色散方程可化简为
\begin{equation}
\omega^2 = \begin{cases}
    k^2 (v_A^2 + c_s^2) \\
    0
\end{cases}
\end{equation}
这表明:慢磁声波消失了,
而快磁声波简化成声速与阿尔芬速度的直接相加。
很多时候我们将这支波快磁声波简称为磁声波。
此时有 $\vec{u}_1 \parallel \vec{k}$,可得流体是可压缩的($\vec{u}_1 \cdot \vec{k} \neq 0$)。

\begin{subequations}\begin{align}
\vec{k} &\perp \hat{\vec{b}} \\
\vec{u}_1 &\parallel \vec{k} \rightarrow \text{流体纵波} \\
\vec{u}_1 &\perp \hat{\vec{b}} \\
\vec{B}_1 &\parallel \hat{\vec{b}} \\
\vec{B}_1 &\perp \vec{k} \rightarrow \text{磁纵波}
\end{align}\end{subequations}

\subsection{各种非理想效应的影响}

\subsubsection{有限电导率}

广义欧姆定律变为:
\begin{equation}
\vec{E} + \vec{u} \times \vec{B} = \eta \vec{J} \tag{\ref{eq:02MHD_欧姆}}
\end{equation}

其扰动形式给出:
\begin{equation}
\vec{E}_1 + \vec{u}_1 \times \vec{B}_0 = \eta \vec{J}_1
\end{equation}
两边用 $\vec{k}$ 叉乘后,代入 $\vec{E}_1$ 与 $\vec{J}_1$ 得
\begin{equation}
\ab( \omega + \im \frac{\eta}{\mu_0} k^2 ) \vec{k} \times \vec{B}_1
= k^2 \big( \vec{u}_1 \times \vec{B}_0 \big) + \Big[ \vec{u}_1 \cdot \big(\vec{k} \times \hat{\vec{b}} \big) \Big] \vec{k}
\end{equation}
代入动量方程得
\begin{equation}
\ab(\omega^2
    - \frac{v_A^2}{1 + \im \frac{\eta}{\mu_0}\frac{k^2}{\omega}} k^2
) \vec{u}_1
= c_s^2 \vec{u}_1 \cdot \vec{k} \vec{k}
- \frac{v_A^2}{1 + \im \frac{\eta}{\mu_0}\frac{k^2}{\omega}} \bigg\{
    \Big[ \vec{u}_1 \cdot \big(\vec{k} \times \hat{\vec{b}} \big) \Big] \vec{k} \times \hat{\vec{b}}
    + k^2 \vec{u}_1 \cdot \hat{\vec{b}} \hat{\vec{b}}
\bigg\}
\end{equation}
可以看到有限电导率的贡献通过阿尔芬速度体现
\begin{equation}
v_A^2 \rightarrow \frac{v_A^2}{1 + \im \frac{\eta}{\mu_0}\frac{k^2}{\omega}}
\end{equation}

对于剪切阿尔芬波,色散关系变为
\begin{equation}
\omega^2 \ab(1 + \im \frac{\eta}{\mu_0}\frac{k^2}{\omega}) = k_\parallel^2 v_A^2
\end{equation}
由于实数电阻引入了虚部,所以频率或者波数必须有虚部,引起耗散,使得波的能量要随时间减少。
考虑波矢 $\vec{k}$ 为复数,则在不同位置振幅不同,适合考察波动在空间中传播的情况。
若考虑频率 $\omega$ 为复数,则在不同时间振幅不同,适合考察波动在时间中传播的情况。

考虑 $\vec{k} \parallel \hat{\vec{b}}$ 的简单情况,若考虑频率 $\omega$ 为复数,则给出
\begin{subequations}\begin{align}
\omega_r^2 &= v_A^2 k^2 - \frac{1}{4} \omega_i^2 \\
\omega_i &= - \frac{\eta}{2 \mu_0} k^2
\end{align}\end{subequations}
其中,频率 $\omega$ 的虚部 $\omega_i$ 也表示波的增长率或阻尼率(damping rate),也记作 $\gamma$。但为了不与绝热指数混淆,本书采用 $\omega_i$。不难发现,$\omega_i \propto \eta, k^2, \lambda^{-2}$,所以短波更容易被阻尼。

同样的,若考虑波矢 $\vec{k}$ 为复数,则给出
\begin{subequations}\begin{align}
\omega^2 &= v_A^2 (k_r^2 - k_i^2) \\
k_i &= \frac{\eta}{2 \mu_0 v_A^2} \omega k_r
\end{align}\end{subequations}
其中,波矢 $\vec{k}$ 的虚部 $k_i$ 也表示波在空间的增长率或阻尼率,其衰减的特征长度 $L = k_i^{-1}$。不难发现,高频波穿透深度短;良导体穿透深度长。

\subsubsection{粘滞效应}

仅考虑粘滞效应对剪切阿尔芬波的影响。在不可压缩条件下,粘滞效应可以简化,只保留第一粘滞系数,则动量方程变为
\begin{equation}
\rho \mdv{\vec{u}}{t} = \vec{J} \times \vec{B} - \grad{p} + w \laplacian{\vec{u}}
\end{equation}

色散关系变为
\begin{equation}
\omega^2 \ab(1 + \im \frac{w}{\rho_0} \frac{k^2}{\omega})
\ab(1 + \im \frac{\eta}{\mu_0} \frac{k^2}{\omega}) = k_\parallel^2 v_A^2
\end{equation}
当 $\dfrac{w}{\rho_0} \dfrac{k^2}{\omega}$、$\dfrac{\eta}{\mu_0} \dfrac{k^2}{\omega}$ 都是小量时,可以再次简化:
\begin{equation}
\omega^2 \ab[1 + \im \ab(\frac{w}{\rho_0} + \frac{\eta}{\mu_0}) \frac{k^2}{\omega}] = k_\parallel^2 v_A^2
\end{equation}
不难发现,只要将 $\dfrac{\eta}{\mu_0} \rightarrow \dfrac{w}{\rho_0} + \dfrac{\eta}{\mu_0}$,就可以完全套用上一节中只有电阻存在时候的结论。这说明在不影响平衡时候的扰动状态下,
粘滞和电阻的角色高度相似。

\subsection{柱坐标系下的阿尔芬波}

上述分析都是立足于最简单的笛卡尔坐标系。
在柱坐标系中平衡量一般来说都是 $r$ 的函数,径向具有不均匀性,从而导致扰动量对径向的依赖不能简单地用 $\eu^{\im k_r r}$ 来表示,需要完整考虑方程中对 $r$ 的偏导。

\subsubsection{柱坐标系下的简单阿尔芬波}

本节考虑 $\vec{k}\parallel\hat{\vec{b}}$ 的情况下最简单的剪切阿尔芬波。
平衡量:
$\vec{u}_0 = \vec{0}$、$\rho_0 = \text{const}$、$p_0 = \text{const}$、$\vec{B}_0 = B_0 \vec{e}_z = \text{const}$、$\vec{J}_0 = \vec{0} $。
扰动量:
$\rho_1 = p_1 = 0$、$\dive{\vec{u}_1} = 0$、$\vec{u}_1 = u_\theta \vec{e}_\theta$、$\vec{B}_1 = B_\theta \vec{e}_\theta$、$\vec{k} = k \vec{e}_z $。
线性化 MHD 方程组:
\begin{subequations}\begin{align}
\rho_0 \pdv{\vec{u}_1}{t} &= \vec{J}_1 \times \vec{B}_0 \\
\vec{E}_1 + \vec{u}_1 \times \vec{B}_0 &= \eta \vec{J}_1 \\
\curl{\vec{E}_1} &= - \pdv{\vec{B}_1}{t} \\
\curl{\vec{B}_1} &= \mu_0 \vec{J}_1
\end{align}\end{subequations}
与笛卡尔坐标系不同,$\nabla$ 不能直接用 $\im\vec{k}$ 代替。
在柱坐标系下,假定扰动量具有如下形式:$f_1 = f_{1,r}(r) \exp{[\im (m \theta + k z - \omega t)]}$。

由不可压缩条件
\begin{equation}\begin{aligned}
\dive{\vec{u}_1} &= \frac{1}{r} \pdv{}{r}(r u_r) + \im \frac{m}{r} u_\theta + \im k_z u_z \\
&= \im \frac{m}{r} u_\theta \\
&= 0
\end{aligned}\end{equation}
得 $m = 0$,$\vec{k} = k \vec{e}_z$。

动量方程
\begin{equation}\begin{aligned}
- \im \omega \mu_0 \rho_0 \vec{u}_1 &=
\big( \curl{\vec{B}_1} \big) \times \vec{B}_0 \\
&= \vec{B}_0 \cdot \grad{\vec{B}_1}
+ \vec{B}_1 \cdot \grad{\vec{B}_0}
- \grad{\big(\vec{B}_1 \cdot \vec{B}_0\big)}
- \big( \curl{\vec{B}_0} \big) \times \vec{B}_1 \\
&= \vec{B}_0 \cdot \grad{\vec{B}_1} \\
&= B_0 \pdv{}{z} \vec{B}_1 \\
&= \im k B_0 \vec{B}_1
\end{aligned}\end{equation}
得
\begin{equation}
u_\theta = - \frac{k}{\omega \mu_0 \rho_0} B_0 B_\theta
\end{equation}

剩余三个方程得到
\begin{equation}
\ab( \pdv{}{t} - \frac{\eta}{\mu_0} \laplacian{} ) \vec{B}_1 = \vec{B}_0 \cdot \grad{\vec{u}_1}
\end{equation}
其中
\begin{equation}\begin{aligned}
\laplacian{\vec{B}_1} &=
\big(\laplacian{B_\theta}\big) \vec{e}_\theta
+ B_\theta \laplacian{\vec{e}_\theta}
+ 2 \big(\grad{B_\theta}\big) \cdot \grad{\vec{e}_\theta} \\
&= \ab[
\frac{1}{r} \pdv{}{r} \ab(r \pdv{B_\theta}{r})
- \ab(\frac{1}{r^2} + k^2) B_\theta
] \vec{e}_\theta
\end{aligned}\end{equation}
\begin{equation}\begin{aligned}
\laplacian{B_\theta} &= \frac{1}{r} \pdv{}{r} \ab(r \pdv{B_\theta}{r})
+ \frac{1}{r^2} \pdv[2]{B_\theta}{\theta}
+ \pdv[2]{B_\theta}{z} \\
&= \frac{1}{r} \pdv{}{r} \ab(r \pdv{B_\theta}{r})
- \ab(\frac{m^2}{r^2} + k_z^2) B_\theta \\
&= \frac{1}{r} \pdv{}{r} \ab(r \pdv{B_\theta}{r})
- k^2 B_\theta
\end{aligned}\end{equation}
\begin{equation}\begin{aligned}
\big(\grad{B_\theta}\big) \cdot \grad{\vec{e}_\theta}
&= \frac{1}{r} \pdv{B_\theta}{\theta} \frac{1}{r} \pdv{\vec{e}_\theta}{\theta} \\
&= - \im \frac{m}{r^2} B_\theta \vec{e}_r \\
&= 0
\end{aligned}\end{equation}
\begin{equation}
\laplacian{\vec{e}_\theta} = - \frac{1}{r^2} \vec{e}_\theta
\end{equation}
得
\begin{equation}
\frac{\eta}{\mu_0 r} \pdv{}{r} \ab(r \pdv{B_\theta}{r})
- \ab[ \frac{\eta}{\mu_0} \ab(\frac{1}{r^2} + k^2) - \im \omega ] B_\theta
= \im k B_0 u_\theta
\end{equation}

代入 $u_\theta$ 得
\begin{equation} \label{eq:04MHDwaves_temp1}
\frac{1}{r} \pdv{}{r} \ab( r \pdv{B_\theta}{r} )
+ \ab( k_c^2 - \frac{1}{r^2} ) B_\theta = 0
\end{equation}
其中
\begin{equation}
k_c^2 = \im \frac{\mu_0}{\eta} \ab( \omega - v_A^2 \frac{k^2}{\omega} ) - k^2
\end{equation}
式 \ref{eq:04MHDwaves_temp1} 为一阶贝塞尔方程的变形,其通解为
\begin{equation}
B_\theta = c_1 J_1(k_c r) + c_2 N_1(k_c r)
\end{equation}
其中,$J_\nu(x)$ 是第一型 Bessel 函数,$N_\nu(x)$ 是第二型 Bessel 函数。
当 $x \to 0$ 时,$N_v(x) \to \infty$,则 $c_2 = 0$。
则解为
\begin{equation}
B_\theta = c_1 J_1(k_c r)
= B_{\theta,k} J_1(k_c r)
\eu^{\im(k z - \omega t)}
\end{equation}
为了得到色散关系,需要利用边界条件。
假定 $r = a$ 处为完全导体壳,则 $\vec{n} \times \vec{E} = 0$。

电场扰动为
\begin{equation}\begin{aligned}
\vec{E}_1 &= \frac{\eta}{\mu_0} \curl{\vec{B}_1}
- \vec{u}_1 \times \vec{B}_0 \\
&= \ab( B_0 u_\theta - \frac{\eta}{\mu_0} \pdv{B_\theta}{z} ) \vec{e}_r
+ \frac{\eta}{\mu_0 r} \pdv{}{r} \big( r B_\theta \big) \vec{e}_z
\end{aligned}\end{equation}
则
\begin{equation}
\frac{\eta}{\mu_0 r} \pdv{}{r} \big( r B_\theta \big) = 0
\end{equation}
由贝塞尔函数性质
\begin{equation}
\odv{}{x} \big[ x^v J_v(x) \big] = x^v J_{v-1}(x)
\end{equation}
得
\begin{equation}
J_0(k_c a) = 0
\end{equation}

则色散关系:
\begin{equation}
\im \frac{\mu_0}{\eta} \ab( \omega - v_A^2 \frac{k^2}{\omega} ) - k^2 = \frac{z_{0,n}^2}{a^2}
\end{equation}
其中,$z_{v,n}$ 为第一型 Bessel 函数 $J_v(x)$ 的第 $n$ 个零点。

可以解得频率为
\begin{equation}
\omega = \sqrt{
    v_A^2 k^2
    - \frac{\eta^2}{4 \mu_0^2} \ab( \frac{z_{0,n}^2}{a^2} + k^2 )^2
} - \im \frac{\eta}{2 \mu_0} \ab( \frac{z_{0,n}^2}{a^2} + k^2 )
\end{equation}
当 $\frac{\eta}{\mu_0} \frac{k^2}{\omega} \ll 1$ 时,有
\begin{equation}
\omega = v_A k - \im \frac{\eta}{2 \mu_0} \ab( \frac{z_{0,n}^2}{a^2} + k^2 )
\end{equation}
可以看出,波动被阻尼。
当 $\eta$ 有限时,波动形成能级分立谱。

\subsubsection{柱坐标系下的压缩阿尔芬波}

本小节考虑流体的压缩效应 $\dive{\vec{u}_1} \neq 0$,但是为了简化问题,仍然只考虑轴对称情形,且只考虑理想 MHD。且考虑 $\vec{u}_1$ 和 $\vec{B}_1$ 的 $r$ 与 $z$ 方向的分量。

\subsection{转动等离子体下的阿尔文波}

在柱坐标系系统中,一个常见的现象是系统存在极向的转动速度。
我们可以通过添加科里奥利力来描述非惯性系中物体的运动方程:
\begin{equation}
\rho \mdv{\vec{u}}{t} = \vec{J} \times \vec{B} - \grad{p}
+ 2 \rho \vec{u} \times \vec{\Omega}
+ \rho \vec{\Omega} \times \big( \vec{\Omega} \times \vec{r} \big)
\end{equation}
其中,$\Omega(r)$ 为转动角频率,
$2 \rho \vec{u} \times \vec{\Omega}$ 为科里奥利力,
$\rho \vec{\Omega} \times ( \vec{\Omega} \times \vec{r} )$ 为惯性离心力。
而惯性离心力只影响平衡量,不影响扰动量。


\begin{equation}
\omega^2 = \ab( 1 \mp 2 \frac{\Omega}{\omega} )^{-1} v_A^2 k^2
\end{equation}???
原来的剪切阿尔文波在科里奥利力的影响下也分裂成相速度 (频率) 高于及低于原来相速度 (频率) 的两支波。

\subsubsection{磁旋转不稳定性}

\begin{subequations}\begin{align}
\pdv{\rho}{t} + \dive{(\rho \vec{u})} &= 0 \\
\rho \pdv{\vec{u}}{t} &= \frac{1}{\mu_0} \big(\curl{\vec{B}}\big) \times \vec{B} - \grad{p} + \rho \vec{g} + \rho w \laplacian{\vec{u}} \\
\pdv{B}{t} &= \curl{\big(\vec{u} \times \vec{B}\big)} \\
\dive{\vec{u}} &= 0
\end{align}\end{subequations}

平衡速度
\begin{equation}
\vec{u}_0 = r \Omega(r) \vec{e}_\theta
\end{equation}

\begin{equation}
\odv{\Omega^2}{\ln{r}} + v_A^2 k^2 < 0
\end{equation}
该判据被称作 Velikhov–Chandrasekhar 不稳定性判据,或者 Balbus-Hawley 判据。

\section{非均匀磁流体中的阿尔芬波}

\subsection{连续谱阿尔芬波}

本小节中,我们只考虑磁场的非均匀性。
为了简化问题,我们采取平板模型:
$\rho_0 = \text{const}, \vec{B}_0 = B_0(x) \vec{e}_z, \vec{k} = k_y \vec{e}_y + k_z \vec{e}_z, f_1 = f_k(x) \exp{[\im(k_y y + k_z z - \omega t)]}$,
且限定不可压缩情况。

由于平衡磁场存在梯度,则压强分布不能是均匀的
\begin{equation}
p_0 + \frac{B_0^2}{2 \mu_0} = \text{const}
\end{equation}
同时也存在压强扰动。

欧姆定律给出
\begin{equation}\begin{aligned}
\pdv{\vec{B}_1}{t} &= \curl{\big( \vec{u}_1 \times \vec{B}_0 \big)} \\
\pdv{\vec{B}_1}{t} &= \vec{B}_0 \cdot \grad{\vec{u}_1} - \vec{u}_1 \cdot \grad{\vec{B}_0} \\
- \im \omega \vec{B}_1 &= \im k_z B_0 \vec{u}_1 - u_x \pdv{B_0}{x} \vec{e}_z
\end{aligned}\end{equation}

动量方程给出
\begin{equation}\begin{aligned}
\rho_0 \mdv{\vec{u}_1}{t} &= \vec{J}_1 \times \vec{B}_0 + \vec{J}_0 \times \vec{B}_1 - \grad{p_1} \\
\rho_0 \pdv{\vec{u}_1}{t} &= \frac{1}{\mu_0} \Big[
    \vec{B}_0 \cdot \grad{\vec{B}_1}
    + \vec{B}_1 \cdot \grad{\vec{B}_0}
    - \grad{\big( \vec{B}_0 \cdot \vec{B}_1 \big) }
\Big] - \grad{p_1} \\
- \im \omega \rho_0 \vec{u}_1 &
= \frac{1}{\mu_0} \ab[
    B_0 k_z \vec{B}_1
    + \big( \vec{B}_1 \cdot \vec{e}_x \big) \pdv{B_0}{x} \vec{e}_z
] - \grad{\ab( p_1 + \frac{\vec{B}_0 \cdot \vec{B}_1}{\mu_0} )}
\end{aligned}\end{equation}

代入 $\vec{B}_1$ 得
\begin{equation}
\ab( \omega^2 - v_A^2 k_\parallel^2 ) \vec{u}_1
= - \im \frac{\omega}{\rho_0} \grad{\ab( p_1 + \frac{\vec{B}_0 \cdot \vec{B}_1}{\mu_0} )}
\end{equation}

这里我们可以用一种稍微不同的手法,利用梯度无旋来消元:
\begin{equation}
\curl{\ab[ \ab( \omega^2 - v_A^2 k_\parallel^2 ) \vec{u}_1 ]} = 0
\end{equation}
考虑上式的 $x$ 分量:
\begin{equation}\begin{aligned}
\vec{e}_x \cdot \curl{\ab[ \ab( \omega^2 - v_A^2 k_\parallel^2 ) \vec{u}_1 ]} &= 0 \\
\dive{\ab[ \ab( \omega^2 - v_A^2 k_\parallel^2 ) \vec{u}_1 \times \vec{e}_x ]} &= 0 \\
\dive{\ab[ \ab( \omega^2 - v_A^2 k_\parallel^2 ) \ab( u_z \vec{e}_y - u_y \vec{e}_z ) ]} &= 0 \\
\im \ab( \omega^2 - v_A^2 k_\parallel^2 ) \ab( u_z k_y - u_y k_z ) &= 0
\end{aligned}\end{equation}
得到
\begin{equation}\label{eq:04MHDwaves_temp2}
u_z k_y - u_y k_z = 0
\end{equation}
类似地,
\begin{equation}\begin{aligned}\label{eq:04MHDwaves_temp3}
\vec{e}_z \cdot \curl{\ab[ \ab( \omega^2 - v_A^2 k_\parallel^2 ) \vec{u}_1 ]} &= 0 \\
\dive{\ab[ \ab( \omega^2 - v_A^2 k_\parallel^2 ) \vec{u}_1 \times \vec{e}_z ]} &= 0 \\
\dive{\ab[ \ab( \omega^2 - v_A^2 k_\parallel^2 ) \ab( u_y \vec{e}_x - u_x \vec{e}_y ) ]} &= 0 \\
\pdv{}{x} \ab[ \ab( \omega^2 - v_A^2 k_\parallel^2 ) u_y ]
- \im \ab( \omega^2 - v_A^2 k_\parallel^2 ) u_x k_y &= 0
\end{aligned}\end{equation}

将式 \ref{eq:04MHDwaves_temp2} 代入不可压缩条件得
\begin{equation}
u_y = \im \frac{k_y}{k^2} \pdv{u_x}{x}
\end{equation}
再将上式代入式 \ref{eq:04MHDwaves_temp3} 得
\begin{equation} \label{eq:04MHDwaves_temp4}
\pdv{}{x} \ab[ \ab( \omega^2 - v_A^2 k_\parallel^2 ) \pdv{u_x}{x} ]
- \ab( \omega^2 - v_A^2 k_\parallel^2 ) u_x k^2
= 0
\end{equation}
定义 $f(x) = \omega^2 - v_A^2 k_\parallel^2$,上式变为:
\begin{equation}
\pdv{}{x} \ab[ f(x) \pdv{u_x}{x} ]
- f(x) u_x k^2
= 0
\end{equation}

可以证明,函数 $f(x)$ 必有零点。证明过程如下:方程两边同时乘以 $u_x$ 的复共轭,然后积分,则有
\begin{equation}\begin{aligned}
\int_{-\infty}^{+\infty} u_x^* \ab[
    \pdv{}{x} \ab( f(x) \pdv{u_x}{x} )
    - f(x) u_x k^2
] \d x &=0 \\
\ab.\ab[ f(x) u_x^* \pdv{u_x}{x} ]|_{-\infty}^{+\infty}
- \int_{-\infty}^{+\infty} \ab[
    f(x) \ab| \pdv{u_x}{x} |^2
    + f(x) |u_x|^2 k^2
] \d x &= 0 \\
\int_{-\infty}^{+\infty}
    f(x) \ab[ \ab| \pdv{u_x}{x} |^2
    + |u_x|^2 k^2 ]
\d x &= 0
\end{aligned}\end{equation}
显然,由于被积分函数中,括号里的项恒大于零,$f(x)$ 在积分区间需有正有负,必有零点。
假如平衡磁场是均匀的,则 $\omega = v_A k_\parallel$ 就是唯一的非平凡解;假如平衡磁场是不均匀的,则形成阿尔芬波连续 $\omega = v_A(x) k_\parallel$。

\subsection{正则奇点与奇异摄动基础}

待续

\subsection{表面阿尔芬波}

表面波处理的是流体表面或者分界面处的情况。由于一般存在密度梯度,则式 \ref{eq:04MHDwaves_temp4} 变为
\begin{equation}
\pdv{}{x} \ab[ \rho_0 \ab( \omega^2 - v_A^2 k_\parallel^2 ) \pdv{u_x}{x} ]
- \rho_0 \ab( \omega^2 - v_A^2 k_\parallel^2 ) u_x k^2
= 0
\end{equation}
在每个区域内部,上式简化成
\begin{equation}
\pdv[2]{u_x}{x} - k^2 u_x = 0
\end{equation}
其通解为
\begin{equation}
u_x = c_1 \eu^{k x} + c_2 \eu^{- k x}
\end{equation}
考虑到远离分界面无穷远处速度扰动为零,再由边界 $x = 0$ 处速度扰动连续,得到
\begin{equation}
u_x = \begin{cases}
    u_{x,c} \eu^{- k x}, & z \geq 0 \\
    u_{x,c} \eu^{k x}  , & z < 0 \\
\end{cases}
\end{equation}

在边界上,需要跨越边界积分:
\begin{equation}\begin{aligned}
\int_{0^-}^{0^+} \ab\{
    \pdv{}{x} \ab[ \rho_0 \ab( \omega^2 - v_A^2 k_\parallel^2 ) \pdv{u_x}{x} ]
    - \rho_0 \ab( \omega^2 - v_A^2 k_\parallel^2 ) u_x k^2
\} \d x &=0 \\
\ab.\ab[
    \rho_0 \ab( \omega^2 - v_A^2 k_\parallel^2 ) \pdv{u_x}{x}
]|_{0^-}^{0^+} &= 0
\end{aligned}\end{equation}
得
\begin{equation}
\omega^2 = \frac{\rho_1 v_{A1}^2 + \rho_2 v_{A2}^2}{\rho_1 + \rho_2} k_\parallel^2
= \frac{B_1^2 + B_2^2}{\mu_0 \ab(\rho_1 + \rho_2)} k_\parallel^2
\end{equation}

\section{环形系统中的测地声模}

本节中,我们介绍环形约束系统中一种独有的静电振荡模式:测地声模(Geodesic acoustic mode, GAM)。
测地声模被视作带状流的高频分支,具有环向对称($n = 0$)和极向近似对称($m \approx 0$)的结构,
因声波沿测地线传播而得名。

出发方程依然是理想的 MHD 方程组,不考虑各向异性或者环向转动。
托卡马克的平衡磁场 $\vec{B} = I(\psi) \grad{\zeta} + \grad{\zeta} \times \grad{\psi} $。

\subsection{环坐标系下的线性化}

定义扰动位移
\begin{equation}
\pdv{\vec{\xi}}{t} = \vec{u}_1
\end{equation}
易得
\begin{equation}
\vec{\xi} = \frac{\im}{\omega} \vec{u}_1
\end{equation}

扰动位移和扰动磁场按如下形式分解:
\begin{subequations}\begin{align}
\vec{\xi} &
= \xi_r \grad{r}
+ \xi_\theta \hat{\vec{b}} \times \grad{r}
+ \xi_\parallel \hat{\vec{b}} \\
\vec{B}_1 &
= B_r \grad{r}
+ B_\theta \hat{\vec{b}} \times \grad{r}
+ B_\parallel \hat{\vec{b}}
\end{align}\end{subequations}

线性化连续性方程得
\begin{equation}
\rho_1 + \grad{\rho_0} \cdot \vec{\xi} + \rho_0 \dive{\vec{\xi}} = 0
\end{equation}

线性化状态方程得
\begin{equation}
p_1 + \grad{p_0} \cdot \vec{\xi} + \gamma p_0 \dive{\vec{\xi}} = 0
\end{equation}

线性化欧姆方程得
\begin{equation}
\vec{B}_1 = \curl{\big( \vec{\xi} \times \vec{B}_0 \big)}
\end{equation}
扰动磁场的分量为
\begin{equation}\begin{aligned}
B_r &= \frac{\vec{B}_1 \cdot \grad{r}}{\Metric^{rr}} \\
&= \frac{1}{\Metric^{rr}} \dive{\ab[ \big(\vec{\xi} \times \vec{B}_0 \big) \times \grad{r} ]} \\
&= \frac{\vec{B}_0}{\Metric^{rr}} \cdot \grad{\big( \xi_r \Metric^{rr} \big)}
\end{aligned}\end{equation}
\begin{equation}\begin{aligned}
B_\parallel &= \vec{B}_1 \cdot \hat{\vec{b}} \\
&= \frac{1}{B_0} \Big\{
\dive{\Big[ \big(\vec{\xi} \times \vec{B}_0 \big) \times \vec{B}_0 \Big]}
+ \big(\vec{\xi} \times \vec{B}_0 \big) \cdot \big(\curl{\vec{B}_0} \big)
\Big\} \\
&= \frac{1}{B_0} \Big[
\dive{\big( \vec{\xi} \cdot \vec{B}_0 \vec{B}_0 - B_0^2 \vec{\xi} \big)}
- \mu_0 \vec{\xi} \cdot \big( \vec{J}_0 \times \vec{B}_0 \big)
\Big] \\
&= - \frac{1}{B} \vec{\xi} \cdot \grad{\big(p_0 + B^2 \big)} - B \dive{\vec{\xi}}
\end{aligned}\end{equation}

待续

测地线曲率
\begin{equation}
    \kappa_g = \vec{\kappa} \cdot (\vec{n} \times \hat{\vec{b}} )
\end{equation}

\subsection{测地声模的色散关系}

忽略径向位移:
\begin{equation}
\xi_r \approx 0
\end{equation}

待续

GAM 的色散关系:
\begin{equation}
\omega^2 = \ab( 2 + \frac{1}{q^2} ) \frac{c_s^2}{R_0^2}
\end{equation}

\section{剪切阿尔芬波的一般描述方法}
